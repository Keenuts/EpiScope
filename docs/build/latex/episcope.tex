%% Generated by Sphinx.
\def\sphinxdocclass{report}
\documentclass[letterpaper,10pt,english]{sphinxmanual}
\ifdefined\pdfpxdimen
   \let\sphinxpxdimen\pdfpxdimen\else\newdimen\sphinxpxdimen
\fi \sphinxpxdimen=.75bp\relax
\ifdefined\pdfimageresolution
    \pdfimageresolution= \numexpr \dimexpr1in\relax/\sphinxpxdimen\relax
\fi
%% let collapsible pdf bookmarks panel have high depth per default
\PassOptionsToPackage{bookmarksdepth=5}{hyperref}

\PassOptionsToPackage{booktabs}{sphinx}
\PassOptionsToPackage{colorrows}{sphinx}

\PassOptionsToPackage{warn}{textcomp}
\usepackage[utf8]{inputenc}
\ifdefined\DeclareUnicodeCharacter
% support both utf8 and utf8x syntaxes
  \ifdefined\DeclareUnicodeCharacterAsOptional
    \def\sphinxDUC#1{\DeclareUnicodeCharacter{"#1}}
  \else
    \let\sphinxDUC\DeclareUnicodeCharacter
  \fi
  \sphinxDUC{00A0}{\nobreakspace}
  \sphinxDUC{2500}{\sphinxunichar{2500}}
  \sphinxDUC{2502}{\sphinxunichar{2502}}
  \sphinxDUC{2514}{\sphinxunichar{2514}}
  \sphinxDUC{251C}{\sphinxunichar{251C}}
  \sphinxDUC{2572}{\textbackslash}
\fi
\usepackage{cmap}
\usepackage[T1]{fontenc}
\usepackage{amsmath,amssymb,amstext}
\usepackage{babel}



\usepackage{tgtermes}
\usepackage{tgheros}
\renewcommand{\ttdefault}{txtt}



\usepackage[Bjarne]{fncychap}
\usepackage{sphinx}

\fvset{fontsize=auto}
\usepackage{geometry}


% Include hyperref last.
\usepackage{hyperref}
% Fix anchor placement for figures with captions.
\usepackage{hypcap}% it must be loaded after hyperref.
% Set up styles of URL: it should be placed after hyperref.
\urlstyle{same}


\usepackage{sphinxmessages}




\title{Episcope}
\date{Apr 21, 2024}
\release{0.1}
\author{Roméo RAMOS\sphinxhyphen{}\sphinxhyphen{}TANGHE, Yosra SAID, Annaelle SARRAZIN, Rachel TECHI, Chloé THIRIET}
\newcommand{\sphinxlogo}{\vbox{}}
\renewcommand{\releasename}{Release}
\makeindex
\begin{document}

\ifdefined\shorthandoff
  \ifnum\catcode`\=\string=\active\shorthandoff{=}\fi
  \ifnum\catcode`\"=\active\shorthandoff{"}\fi
\fi

\pagestyle{empty}
\sphinxmaketitle
\pagestyle{plain}
\sphinxtableofcontents
\pagestyle{normal}
\phantomsection\label{\detokenize{index::doc}}
\noindent{\hspace*{\fill}\sphinxincludegraphics[width=500\sphinxpxdimen,height=110\sphinxpxdimen]{{Logo_final}.png}\hspace*{\fill}}

\sphinxAtStartPar
\sphinxstylestrong{EpiScope} is a GUI featuring ergonomic tools for annotating epileptic seizure videos.
These tools enable practitioners to note directly on their patients’ epileptic seizure videos the various symptoms that appear, thanks to a pre\sphinxhyphen{}configured symptom semiology. This interface generates a .txt text file listing all the symptoms occurring during the seizure in chronological order, as well as a timeline illustrating the patient’s epileptic seizure. The timeline follows a temporal axis (identical to that of the seizure video) and indicates the moment of onset and end of each symptom. Practitioners must also be able to modify the .txt file and the frieze in the event of an oversight or readjustment.

\sphinxstepscope



\sphinxAtStartPar
Episcope is a GUI featuring ergonomic tools for annotating epileptic seizure videos.

\sphinxAtStartPar
These tools enable practitioners to note directly on their patients’ epileptic seizure videos the various symptoms that appear, thanks to a pre\sphinxhyphen{}configured symptom semiology.
This interface generates a .txt text file listing all the symptoms occurring during the seizure in chronological order, as well as a timeline illustrating the patient’s
epileptic seizure. The timeline follows a temporal axis (identical to that of the seizure video) and indicates the moment of onset and end of each symptom. Practitioners must
also be able to modify the .txt file and the frieze in the event of an oversight or readjustment.


\chapter{Examples}
\label{\detokenize{GUI:examples}}
\noindent{\hspace*{\fill}\sphinxincludegraphics[width=500\sphinxpxdimen,height=110\sphinxpxdimen]{{interface_vide}.png}\hspace*{\fill}}

\sphinxAtStartPar
The GUI is divided in 3 parts with dropdown menus, a video player and a current symptoms zone.
Check out the user manual to find out more about the usage of Episcope GUI

\sphinxstepscope


\chapter{General Interface}
\label{\detokenize{general_interface:module-general_interface}}\label{\detokenize{general_interface:general-interface}}\label{\detokenize{general_interface::doc}}\index{module@\spxentry{module}!general\_interface@\spxentry{general\_interface}}\index{general\_interface@\spxentry{general\_interface}!module@\spxentry{module}}
\sphinxAtStartPar
Episcope general interface containing
\begin{itemize}
\item {} \begin{description}
\sphinxlineitem{video player :}\begin{itemize}
\item {} 
\sphinxAtStartPar
video playback

\item {} 
\sphinxAtStartPar
play/pause, skip\textgreater{}\textgreater{}, skip\textless{}\textless{}, review buttons

\item {} 
\sphinxAtStartPar
advance and rewind by 1s and pause

\item {} 
\sphinxAtStartPar
synchronised sound

\item {} 
\sphinxAtStartPar
good speed

\item {} 
\sphinxAtStartPar
progression bar sychronised

\end{itemize}

\end{description}

\item {} \begin{description}
\sphinxlineitem{annotations :}\begin{itemize}
\item {} 
\sphinxAtStartPar
menus based on exel file (xls)

\item {} 
\sphinxAtStartPar
pre\sphinxhyphen{}load symptoms

\item {} 
\sphinxAtStartPar
cascading menus on left

\item {} 
\sphinxAtStartPar
search bar

\item {} 
\sphinxAtStartPar
correct initialization of symptoms

\item {} 
\sphinxAtStartPar
retrieve symptoms from a list

\item {} 
\sphinxAtStartPar
retrieve start and end times

\item {} 
\sphinxAtStartPar
display symptoms on the right

\item {} 
\sphinxAtStartPar
scrollable symptoms

\item {} 
\sphinxAtStartPar
pop\sphinxhyphen{}up to modify symptoms

\end{itemize}

\end{description}

\item {} \begin{description}
\sphinxlineitem{files :}\begin{itemize}
\item {} 
\sphinxAtStartPar
generate frieze

\item {} 
\sphinxAtStartPar
generate a report

\item {} 
\sphinxAtStartPar
generate a symptom file

\end{itemize}

\end{description}

\end{itemize}

\sphinxAtStartPar
modification of the general design and buttons of the drop\sphinxhyphen{}down menus and the video progress management buttons
everything is in english
configuration of the progress bar(state of the cursor:pressed, not pressed)
version : 0.10.1
\index{FriseSymptomes (class in general\_interface)@\spxentry{FriseSymptomes}\spxextra{class in general\_interface}}

\begin{fulllineitems}
\phantomsection\label{\detokenize{general_interface:general_interface.FriseSymptomes}}
\pysigstartsignatures
\pysiglinewithargsret{\sphinxbfcode{\sphinxupquote{class\DUrole{w}{ }}}\sphinxcode{\sphinxupquote{general\_interface.}}\sphinxbfcode{\sphinxupquote{FriseSymptomes}}}{\sphinxparam{\DUrole{n}{InterfaceGenerale}}\sphinxparamcomma \sphinxparam{\DUrole{n}{MenuDeroulant}}}{}
\pysigstopsignatures
\sphinxAtStartPar
Bases: \sphinxcode{\sphinxupquote{object}}

\sphinxAtStartPar
Class used to generate a timeline summarising all the symptoms present during an epileptic seizure.
\index{interfaceGenerale (general\_interface.FriseSymptomes attribute)@\spxentry{interfaceGenerale}\spxextra{general\_interface.FriseSymptomes attribute}}

\begin{fulllineitems}
\phantomsection\label{\detokenize{general_interface:general_interface.FriseSymptomes.interfaceGenerale}}
\pysigstartsignatures
\pysigline{\sphinxbfcode{\sphinxupquote{interfaceGenerale}}}
\pysigstopsignatures
\sphinxAtStartPar
the general interface with the symptoms list
\begin{quote}\begin{description}
\sphinxlineitem{Type}
\sphinxAtStartPar
{\hyperref[\detokenize{general_interface:general_interface.InterfaceGenerale}]{\sphinxcrossref{\sphinxcode{\sphinxupquote{InterfaceGenerale}}}}}

\end{description}\end{quote}

\end{fulllineitems}

\index{MenuDeroulant (general\_interface.FriseSymptomes attribute)@\spxentry{MenuDeroulant}\spxextra{general\_interface.FriseSymptomes attribute}}

\begin{fulllineitems}
\phantomsection\label{\detokenize{general_interface:general_interface.FriseSymptomes.MenuDeroulant}}
\pysigstartsignatures
\pysigline{\sphinxbfcode{\sphinxupquote{MenuDeroulant}}}
\pysigstopsignatures
\sphinxAtStartPar
dropdown menus
\begin{quote}\begin{description}
\sphinxlineitem{Type}
\sphinxAtStartPar
{\hyperref[\detokenize{general_interface:general_interface.Menu_symptomes}]{\sphinxcrossref{\sphinxcode{\sphinxupquote{Menu\_symptomes}}}}}

\end{description}\end{quote}

\end{fulllineitems}

\index{afficher() (general\_interface.FriseSymptomes method)@\spxentry{afficher()}\spxextra{general\_interface.FriseSymptomes method}}

\begin{fulllineitems}
\phantomsection\label{\detokenize{general_interface:general_interface.FriseSymptomes.afficher}}
\pysigstartsignatures
\pysiglinewithargsret{\sphinxbfcode{\sphinxupquote{afficher}}}{}{}
\pysigstopsignatures
\sphinxAtStartPar
Retrieves the list of symptoms instantiated in the InterfaceGenerale class’s update\_right\_panel function.

\sphinxAtStartPar
This method retrieves the list of symptoms stored in the InterfaceGenerale class instance’s ListeSymptomes attribute.
It then processes each symptom in the list, extracting relevant information such as name, start time, end time,
ID, lateralization, topography, orientation, additional attributes, and comments. This information is then formatted
and sorted based on the start time of each symptom.
\begin{quote}\begin{description}
\sphinxlineitem{Returns}
\sphinxAtStartPar
None

\end{description}\end{quote}

\end{fulllineitems}


\end{fulllineitems}

\index{InterfaceGenerale (class in general\_interface)@\spxentry{InterfaceGenerale}\spxextra{class in general\_interface}}

\begin{fulllineitems}
\phantomsection\label{\detokenize{general_interface:general_interface.InterfaceGenerale}}
\pysigstartsignatures
\pysiglinewithargsret{\sphinxbfcode{\sphinxupquote{class\DUrole{w}{ }}}\sphinxcode{\sphinxupquote{general\_interface.}}\sphinxbfcode{\sphinxupquote{InterfaceGenerale}}}{\sphinxparam{\DUrole{n}{fenetre}}}{}
\pysigstopsignatures
\sphinxAtStartPar
Bases: \sphinxcode{\sphinxupquote{object}}

\sphinxAtStartPar
General interface class, which provides the overall look and feel of the graphical interface.

\sphinxAtStartPar
It is represented in the form of a window and allows the other classes to be integrated into it.
It calls the LecteurVideo, FriseSymptomes and Menu\_symptoms classes.
\index{fenetre (general\_interface.InterfaceGenerale attribute)@\spxentry{fenetre}\spxextra{general\_interface.InterfaceGenerale attribute}}

\begin{fulllineitems}
\phantomsection\label{\detokenize{general_interface:general_interface.InterfaceGenerale.fenetre}}
\pysigstartsignatures
\pysigline{\sphinxbfcode{\sphinxupquote{fenetre}}}
\pysigstopsignatures
\sphinxAtStartPar
Represents the main window of the graphical interface.

\end{fulllineitems}

\index{cap (general\_interface.InterfaceGenerale attribute)@\spxentry{cap}\spxextra{general\_interface.InterfaceGenerale attribute}}

\begin{fulllineitems}
\phantomsection\label{\detokenize{general_interface:general_interface.InterfaceGenerale.cap}}
\pysigstartsignatures
\pysigline{\sphinxbfcode{\sphinxupquote{cap}}}
\pysigstopsignatures
\sphinxAtStartPar
Represents the video capture object.

\end{fulllineitems}

\index{lec\_video (general\_interface.InterfaceGenerale attribute)@\spxentry{lec\_video}\spxextra{general\_interface.InterfaceGenerale attribute}}

\begin{fulllineitems}
\phantomsection\label{\detokenize{general_interface:general_interface.InterfaceGenerale.lec_video}}
\pysigstartsignatures
\pysigline{\sphinxbfcode{\sphinxupquote{lec\_video}}}
\pysigstopsignatures
\sphinxAtStartPar
An instance of the LecteurVideo class.

\end{fulllineitems}

\index{frise (general\_interface.InterfaceGenerale attribute)@\spxentry{frise}\spxextra{general\_interface.InterfaceGenerale attribute}}

\begin{fulllineitems}
\phantomsection\label{\detokenize{general_interface:general_interface.InterfaceGenerale.frise}}
\pysigstartsignatures
\pysigline{\sphinxbfcode{\sphinxupquote{frise}}}
\pysigstopsignatures
\sphinxAtStartPar
An instance of the FriseSymptomes class.

\end{fulllineitems}

\index{set\_end\_time\_mode (general\_interface.InterfaceGenerale attribute)@\spxentry{set\_end\_time\_mode}\spxextra{general\_interface.InterfaceGenerale attribute}}

\begin{fulllineitems}
\phantomsection\label{\detokenize{general_interface:general_interface.InterfaceGenerale.set_end_time_mode}}
\pysigstartsignatures
\pysigline{\sphinxbfcode{\sphinxupquote{set\_end\_time\_mode}}}
\pysigstopsignatures
\sphinxAtStartPar
A boolean variable indicating whether the end time mode is set.

\end{fulllineitems}

\index{current\_symptom (general\_interface.InterfaceGenerale attribute)@\spxentry{current\_symptom}\spxextra{general\_interface.InterfaceGenerale attribute}}

\begin{fulllineitems}
\phantomsection\label{\detokenize{general_interface:general_interface.InterfaceGenerale.current_symptom}}
\pysigstartsignatures
\pysigline{\sphinxbfcode{\sphinxupquote{current\_symptom}}}
\pysigstopsignatures
\sphinxAtStartPar
Represents the current symptom being processed.

\end{fulllineitems}

\index{ListeSymptomes (general\_interface.InterfaceGenerale attribute)@\spxentry{ListeSymptomes}\spxextra{general\_interface.InterfaceGenerale attribute}}

\begin{fulllineitems}
\phantomsection\label{\detokenize{general_interface:general_interface.InterfaceGenerale.ListeSymptomes}}
\pysigstartsignatures
\pysigline{\sphinxbfcode{\sphinxupquote{ListeSymptomes}}}
\pysigstopsignatures
\sphinxAtStartPar
A list containing symptom objects, initially empty.

\end{fulllineitems}

\index{theme (general\_interface.InterfaceGenerale attribute)@\spxentry{theme}\spxextra{general\_interface.InterfaceGenerale attribute}}

\begin{fulllineitems}
\phantomsection\label{\detokenize{general_interface:general_interface.InterfaceGenerale.theme}}
\pysigstartsignatures
\pysigline{\sphinxbfcode{\sphinxupquote{theme}}}
\pysigstopsignatures
\sphinxAtStartPar
A list containing color theme values.

\end{fulllineitems}

\index{menu\_bar (general\_interface.InterfaceGenerale attribute)@\spxentry{menu\_bar}\spxextra{general\_interface.InterfaceGenerale attribute}}

\begin{fulllineitems}
\phantomsection\label{\detokenize{general_interface:general_interface.InterfaceGenerale.menu_bar}}
\pysigstartsignatures
\pysigline{\sphinxbfcode{\sphinxupquote{menu\_bar}}}
\pysigstopsignatures
\sphinxAtStartPar
A menu bar for the interface.

\end{fulllineitems}

\index{menu\_open (general\_interface.InterfaceGenerale attribute)@\spxentry{menu\_open}\spxextra{general\_interface.InterfaceGenerale attribute}}

\begin{fulllineitems}
\phantomsection\label{\detokenize{general_interface:general_interface.InterfaceGenerale.menu_open}}
\pysigstartsignatures
\pysigline{\sphinxbfcode{\sphinxupquote{menu\_open}}}
\pysigstopsignatures
\sphinxAtStartPar
A menu for opening different options like video and symptoms.

\end{fulllineitems}

\index{menu\_save (general\_interface.InterfaceGenerale attribute)@\spxentry{menu\_save}\spxextra{general\_interface.InterfaceGenerale attribute}}

\begin{fulllineitems}
\phantomsection\label{\detokenize{general_interface:general_interface.InterfaceGenerale.menu_save}}
\pysigstartsignatures
\pysigline{\sphinxbfcode{\sphinxupquote{menu\_save}}}
\pysigstopsignatures
\sphinxAtStartPar
A menu for saving options like symptoms, report, and timeline.

\end{fulllineitems}

\index{clic\_x (general\_interface.InterfaceGenerale attribute)@\spxentry{clic\_x}\spxextra{general\_interface.InterfaceGenerale attribute}}

\begin{fulllineitems}
\phantomsection\label{\detokenize{general_interface:general_interface.InterfaceGenerale.clic_x}}
\pysigstartsignatures
\pysigline{\sphinxbfcode{\sphinxupquote{clic\_x}}}
\pysigstopsignatures
\sphinxAtStartPar
The x\sphinxhyphen{}coordinate of the click event.

\end{fulllineitems}

\index{clic\_y (general\_interface.InterfaceGenerale attribute)@\spxentry{clic\_y}\spxextra{general\_interface.InterfaceGenerale attribute}}

\begin{fulllineitems}
\phantomsection\label{\detokenize{general_interface:general_interface.InterfaceGenerale.clic_y}}
\pysigstartsignatures
\pysigline{\sphinxbfcode{\sphinxupquote{clic\_y}}}
\pysigstopsignatures
\sphinxAtStartPar
The y\sphinxhyphen{}coordinate of the click event.

\end{fulllineitems}

\index{frame\_left (general\_interface.InterfaceGenerale attribute)@\spxentry{frame\_left}\spxextra{general\_interface.InterfaceGenerale attribute}}

\begin{fulllineitems}
\phantomsection\label{\detokenize{general_interface:general_interface.InterfaceGenerale.frame_left}}
\pysigstartsignatures
\pysigline{\sphinxbfcode{\sphinxupquote{frame\_left}}}
\pysigstopsignatures
\sphinxAtStartPar
Represents the left frame of the interface.

\end{fulllineitems}

\index{frame\_middle (general\_interface.InterfaceGenerale attribute)@\spxentry{frame\_middle}\spxextra{general\_interface.InterfaceGenerale attribute}}

\begin{fulllineitems}
\phantomsection\label{\detokenize{general_interface:general_interface.InterfaceGenerale.frame_middle}}
\pysigstartsignatures
\pysigline{\sphinxbfcode{\sphinxupquote{frame\_middle}}}
\pysigstopsignatures
\sphinxAtStartPar
Represents the middle frame of the interface.

\end{fulllineitems}

\index{frame\_right (general\_interface.InterfaceGenerale attribute)@\spxentry{frame\_right}\spxextra{general\_interface.InterfaceGenerale attribute}}

\begin{fulllineitems}
\phantomsection\label{\detokenize{general_interface:general_interface.InterfaceGenerale.frame_right}}
\pysigstartsignatures
\pysigline{\sphinxbfcode{\sphinxupquote{frame\_right}}}
\pysigstopsignatures
\sphinxAtStartPar
Represents the right frame of the interface.

\end{fulllineitems}

\index{scrollable1 (general\_interface.InterfaceGenerale attribute)@\spxentry{scrollable1}\spxextra{general\_interface.InterfaceGenerale attribute}}

\begin{fulllineitems}
\phantomsection\label{\detokenize{general_interface:general_interface.InterfaceGenerale.scrollable1}}
\pysigstartsignatures
\pysigline{\sphinxbfcode{\sphinxupquote{scrollable1}}}
\pysigstopsignatures
\sphinxAtStartPar
A scrollable frame for the right panel.

\end{fulllineitems}

\index{text\_output (general\_interface.InterfaceGenerale attribute)@\spxentry{text\_output}\spxextra{general\_interface.InterfaceGenerale attribute}}

\begin{fulllineitems}
\phantomsection\label{\detokenize{general_interface:general_interface.InterfaceGenerale.text_output}}
\pysigstartsignatures
\pysigline{\sphinxbfcode{\sphinxupquote{text\_output}}}
\pysigstopsignatures
\sphinxAtStartPar
A scrollable frame for displaying text output.

\end{fulllineitems}

\index{frame\_CTkButton (general\_interface.InterfaceGenerale attribute)@\spxentry{frame\_CTkButton}\spxextra{general\_interface.InterfaceGenerale attribute}}

\begin{fulllineitems}
\phantomsection\label{\detokenize{general_interface:general_interface.InterfaceGenerale.frame_CTkButton}}
\pysigstartsignatures
\pysigline{\sphinxbfcode{\sphinxupquote{frame\_CTkButton}}}
\pysigstopsignatures
\sphinxAtStartPar
A frame for buttons in the middle frame.

\end{fulllineitems}

\index{bouton\_revoir (general\_interface.InterfaceGenerale attribute)@\spxentry{bouton\_revoir}\spxextra{general\_interface.InterfaceGenerale attribute}}

\begin{fulllineitems}
\phantomsection\label{\detokenize{general_interface:general_interface.InterfaceGenerale.bouton_revoir}}
\pysigstartsignatures
\pysigline{\sphinxbfcode{\sphinxupquote{bouton\_revoir}}}
\pysigstopsignatures
\sphinxAtStartPar
Button for rewinding the video.

\end{fulllineitems}

\index{bouton\_reculer (general\_interface.InterfaceGenerale attribute)@\spxentry{bouton\_reculer}\spxextra{general\_interface.InterfaceGenerale attribute}}

\begin{fulllineitems}
\phantomsection\label{\detokenize{general_interface:general_interface.InterfaceGenerale.bouton_reculer}}
\pysigstartsignatures
\pysigline{\sphinxbfcode{\sphinxupquote{bouton\_reculer}}}
\pysigstopsignatures
\sphinxAtStartPar
Button for moving the video backward.

\end{fulllineitems}

\index{bouton\_play\_pause (general\_interface.InterfaceGenerale attribute)@\spxentry{bouton\_play\_pause}\spxextra{general\_interface.InterfaceGenerale attribute}}

\begin{fulllineitems}
\phantomsection\label{\detokenize{general_interface:general_interface.InterfaceGenerale.bouton_play_pause}}
\pysigstartsignatures
\pysigline{\sphinxbfcode{\sphinxupquote{bouton\_play\_pause}}}
\pysigstopsignatures
\sphinxAtStartPar
Button for playing or pausing the video.

\end{fulllineitems}

\index{bouton\_avancer (general\_interface.InterfaceGenerale attribute)@\spxentry{bouton\_avancer}\spxextra{general\_interface.InterfaceGenerale attribute}}

\begin{fulllineitems}
\phantomsection\label{\detokenize{general_interface:general_interface.InterfaceGenerale.bouton_avancer}}
\pysigstartsignatures
\pysigline{\sphinxbfcode{\sphinxupquote{bouton\_avancer}}}
\pysigstopsignatures
\sphinxAtStartPar
Button for moving the video forward.

\end{fulllineitems}

\index{bouton\_nul (general\_interface.InterfaceGenerale attribute)@\spxentry{bouton\_nul}\spxextra{general\_interface.InterfaceGenerale attribute}}

\begin{fulllineitems}
\phantomsection\label{\detokenize{general_interface:general_interface.InterfaceGenerale.bouton_nul}}
\pysigstartsignatures
\pysigline{\sphinxbfcode{\sphinxupquote{bouton\_nul}}}
\pysigstopsignatures
\sphinxAtStartPar
A dummy button.

\end{fulllineitems}

\index{frame\_frise (general\_interface.InterfaceGenerale attribute)@\spxentry{frame\_frise}\spxextra{general\_interface.InterfaceGenerale attribute}}

\begin{fulllineitems}
\phantomsection\label{\detokenize{general_interface:general_interface.InterfaceGenerale.frame_frise}}
\pysigstartsignatures
\pysigline{\sphinxbfcode{\sphinxupquote{frame\_frise}}}
\pysigstopsignatures
\sphinxAtStartPar
A frame for displaying the symptom timeline.

\end{fulllineitems}

\index{bouton\_frise (general\_interface.InterfaceGenerale attribute)@\spxentry{bouton\_frise}\spxextra{general\_interface.InterfaceGenerale attribute}}

\begin{fulllineitems}
\phantomsection\label{\detokenize{general_interface:general_interface.InterfaceGenerale.bouton_frise}}
\pysigstartsignatures
\pysigline{\sphinxbfcode{\sphinxupquote{bouton\_frise}}}
\pysigstopsignatures
\sphinxAtStartPar
Button for displaying the timeline.

\end{fulllineitems}

\index{bouton\_save (general\_interface.InterfaceGenerale attribute)@\spxentry{bouton\_save}\spxextra{general\_interface.InterfaceGenerale attribute}}

\begin{fulllineitems}
\phantomsection\label{\detokenize{general_interface:general_interface.InterfaceGenerale.bouton_save}}
\pysigstartsignatures
\pysigline{\sphinxbfcode{\sphinxupquote{bouton\_save}}}
\pysigstopsignatures
\sphinxAtStartPar
Button for saving the report.

\end{fulllineitems}

\index{label\_temps (general\_interface.InterfaceGenerale attribute)@\spxentry{label\_temps}\spxextra{general\_interface.InterfaceGenerale attribute}}

\begin{fulllineitems}
\phantomsection\label{\detokenize{general_interface:general_interface.InterfaceGenerale.label_temps}}
\pysigstartsignatures
\pysigline{\sphinxbfcode{\sphinxupquote{label\_temps}}}
\pysigstopsignatures
\sphinxAtStartPar
Label for displaying elapsed time and total time.

\end{fulllineitems}

\index{progress\_slider (general\_interface.InterfaceGenerale attribute)@\spxentry{progress\_slider}\spxextra{general\_interface.InterfaceGenerale attribute}}

\begin{fulllineitems}
\phantomsection\label{\detokenize{general_interface:general_interface.InterfaceGenerale.progress_slider}}
\pysigstartsignatures
\pysigline{\sphinxbfcode{\sphinxupquote{progress\_slider}}}
\pysigstopsignatures
\sphinxAtStartPar
A slider for manual video progress.

\end{fulllineitems}

\index{canvas (general\_interface.InterfaceGenerale attribute)@\spxentry{canvas}\spxextra{general\_interface.InterfaceGenerale attribute}}

\begin{fulllineitems}
\phantomsection\label{\detokenize{general_interface:general_interface.InterfaceGenerale.canvas}}
\pysigstartsignatures
\pysigline{\sphinxbfcode{\sphinxupquote{canvas}}}
\pysigstopsignatures
\sphinxAtStartPar
Canvas for displaying the video.

\end{fulllineitems}

\index{zone\_text (general\_interface.InterfaceGenerale attribute)@\spxentry{zone\_text}\spxextra{general\_interface.InterfaceGenerale attribute}}

\begin{fulllineitems}
\phantomsection\label{\detokenize{general_interface:general_interface.InterfaceGenerale.zone_text}}
\pysigstartsignatures
\pysigline{\sphinxbfcode{\sphinxupquote{zone\_text}}}
\pysigstopsignatures
\sphinxAtStartPar
A text area for comments.

\end{fulllineitems}

\index{get\_current\_video\_time() (general\_interface.InterfaceGenerale method)@\spxentry{get\_current\_video\_time()}\spxextra{general\_interface.InterfaceGenerale method}}

\begin{fulllineitems}
\phantomsection\label{\detokenize{general_interface:general_interface.InterfaceGenerale.get_current_video_time}}
\pysigstartsignatures
\pysiglinewithargsret{\sphinxbfcode{\sphinxupquote{get\_current\_video\_time}}}{}{}
\pysigstopsignatures
\sphinxAtStartPar
Get the current time of the video.
\begin{quote}\begin{description}
\sphinxlineitem{Returns}
\sphinxAtStartPar
\begin{description}
\sphinxlineitem{A string representing the current time of the video in the format “HH:MM:SS”.}
\sphinxAtStartPar
Returns “00:00:00” if the video is not opened.

\end{description}


\sphinxlineitem{Return type}
\sphinxAtStartPar
str

\end{description}\end{quote}

\end{fulllineitems}

\index{get\_video\_duration() (general\_interface.InterfaceGenerale method)@\spxentry{get\_video\_duration()}\spxextra{general\_interface.InterfaceGenerale method}}

\begin{fulllineitems}
\phantomsection\label{\detokenize{general_interface:general_interface.InterfaceGenerale.get_video_duration}}
\pysigstartsignatures
\pysiglinewithargsret{\sphinxbfcode{\sphinxupquote{get\_video\_duration}}}{}{}
\pysigstopsignatures
\sphinxAtStartPar
Get the total duration of the video.
\begin{quote}\begin{description}
\sphinxlineitem{Returns}
\sphinxAtStartPar
\begin{description}
\sphinxlineitem{The total duration of the video in seconds.}
\sphinxAtStartPar
Returns 0 if the video is not opened.

\end{description}


\sphinxlineitem{Return type}
\sphinxAtStartPar
int

\end{description}\end{quote}

\end{fulllineitems}

\index{lire\_fichier() (general\_interface.InterfaceGenerale method)@\spxentry{lire\_fichier()}\spxextra{general\_interface.InterfaceGenerale method}}

\begin{fulllineitems}
\phantomsection\label{\detokenize{general_interface:general_interface.InterfaceGenerale.lire_fichier}}
\pysigstartsignatures
\pysiglinewithargsret{\sphinxbfcode{\sphinxupquote{lire\_fichier}}}{\sphinxparam{\DUrole{n}{nom\_fichier}}}{}
\pysigstopsignatures
\sphinxAtStartPar
Opens a text file, reads it, and saves the information in a list.
\begin{quote}\begin{description}
\sphinxlineitem{Parameters}
\sphinxAtStartPar
\sphinxstyleliteralstrong{\sphinxupquote{nom\_fichier}} (\sphinxstyleliteralemphasis{\sphinxupquote{str}}) \textendash{} The path to the text file to be read.

\sphinxlineitem{Returns}
\sphinxAtStartPar
A list containing all the lines read from the file.

\sphinxlineitem{Return type}
\sphinxAtStartPar
list

\sphinxlineitem{Raises}
\sphinxAtStartPar
\sphinxstyleliteralstrong{\sphinxupquote{FileNotFoundError}} \textendash{} If the specified file is not found.

\end{description}\end{quote}

\end{fulllineitems}

\index{load\_symptoms() (general\_interface.InterfaceGenerale method)@\spxentry{load\_symptoms()}\spxextra{general\_interface.InterfaceGenerale method}}

\begin{fulllineitems}
\phantomsection\label{\detokenize{general_interface:general_interface.InterfaceGenerale.load_symptoms}}
\pysigstartsignatures
\pysiglinewithargsret{\sphinxbfcode{\sphinxupquote{load\_symptoms}}}{}{}
\pysigstopsignatures
\sphinxAtStartPar
Loads a list of symptoms from a text file.

\sphinxAtStartPar
This function prompts the user to select a text file containing symptom data. It then reads the symptom data from the file
using the read\_symptoms function from the load\_symptomes module. For each symptom read from the file, it updates the right panel of
the interface with the symptom attributes.
\begin{quote}\begin{description}
\sphinxlineitem{Raises}\begin{itemize}
\item {} 
\sphinxAtStartPar
\sphinxstyleliteralstrong{\sphinxupquote{FileNotFoundError}} \textendash{} If the selected file is not found.

\item {} 
\sphinxAtStartPar
\sphinxstyleliteralstrong{\sphinxupquote{Exception}} \textendash{} If an unexpected error occurs during the file reading process.

\end{itemize}

\end{description}\end{quote}

\end{fulllineitems}

\index{ouvrir\_video() (general\_interface.InterfaceGenerale method)@\spxentry{ouvrir\_video()}\spxextra{general\_interface.InterfaceGenerale method}}

\begin{fulllineitems}
\phantomsection\label{\detokenize{general_interface:general_interface.InterfaceGenerale.ouvrir_video}}
\pysigstartsignatures
\pysiglinewithargsret{\sphinxbfcode{\sphinxupquote{ouvrir\_video}}}{}{}
\pysigstopsignatures
\sphinxAtStartPar
Opens the video by calling the Lecteur\_video class

\end{fulllineitems}

\index{ouvrir\_video\_noire() (general\_interface.InterfaceGenerale method)@\spxentry{ouvrir\_video\_noire()}\spxextra{general\_interface.InterfaceGenerale method}}

\begin{fulllineitems}
\phantomsection\label{\detokenize{general_interface:general_interface.InterfaceGenerale.ouvrir_video_noire}}
\pysigstartsignatures
\pysiglinewithargsret{\sphinxbfcode{\sphinxupquote{ouvrir\_video\_noire}}}{}{}
\pysigstopsignatures
\sphinxAtStartPar
Opens the black\_video by calling the Lecteur\_video class

\end{fulllineitems}

\index{rapport() (general\_interface.InterfaceGenerale method)@\spxentry{rapport()}\spxextra{general\_interface.InterfaceGenerale method}}

\begin{fulllineitems}
\phantomsection\label{\detokenize{general_interface:general_interface.InterfaceGenerale.rapport}}
\pysigstartsignatures
\pysiglinewithargsret{\sphinxbfcode{\sphinxupquote{rapport}}}{}{}
\pysigstopsignatures
\sphinxAtStartPar
Writes the report of the seizure

\end{fulllineitems}

\index{sauvegarde() (general\_interface.InterfaceGenerale method)@\spxentry{sauvegarde()}\spxextra{general\_interface.InterfaceGenerale method}}

\begin{fulllineitems}
\phantomsection\label{\detokenize{general_interface:general_interface.InterfaceGenerale.sauvegarde}}
\pysigstartsignatures
\pysiglinewithargsret{\sphinxbfcode{\sphinxupquote{sauvegarde}}}{}{}
\pysigstopsignatures
\sphinxAtStartPar
Saves the list of symptoms to a text file.
It calls the save class for this purpose.

\end{fulllineitems}

\index{supprimer() (general\_interface.InterfaceGenerale method)@\spxentry{supprimer()}\spxextra{general\_interface.InterfaceGenerale method}}

\begin{fulllineitems}
\phantomsection\label{\detokenize{general_interface:general_interface.InterfaceGenerale.supprimer}}
\pysigstartsignatures
\pysiglinewithargsret{\sphinxbfcode{\sphinxupquote{supprimer}}}{\sphinxparam{\DUrole{n}{selection}}\sphinxparamcomma \sphinxparam{\DUrole{n}{container}}}{}
\pysigstopsignatures
\sphinxAtStartPar
Remove a symptom from the list of symptoms and destroy its display container.
\begin{quote}\begin{description}
\sphinxlineitem{Parameters}\begin{itemize}
\item {} 
\sphinxAtStartPar
\sphinxstyleliteralstrong{\sphinxupquote{selection}} \textendash{} The symptom object to be removed.

\item {} 
\sphinxAtStartPar
\sphinxstyleliteralstrong{\sphinxupquote{container}} \textendash{} The container containing the display elements of the symptom.

\end{itemize}

\sphinxlineitem{Returns}
\sphinxAtStartPar
None

\end{description}\end{quote}

\end{fulllineitems}

\index{text\_sympt() (general\_interface.InterfaceGenerale method)@\spxentry{text\_sympt()}\spxextra{general\_interface.InterfaceGenerale method}}

\begin{fulllineitems}
\phantomsection\label{\detokenize{general_interface:general_interface.InterfaceGenerale.text_sympt}}
\pysigstartsignatures
\pysiglinewithargsret{\sphinxbfcode{\sphinxupquote{text\_sympt}}}{\sphinxparam{\DUrole{n}{sympt}}}{}
\pysigstopsignatures
\sphinxAtStartPar
Creates the text to display in the right panel.
\begin{quote}\begin{description}
\sphinxlineitem{Parameters}
\sphinxAtStartPar
\sphinxstyleliteralstrong{\sphinxupquote{sympt}} ({\hyperref[\detokenize{annotation:annotation.class_symptome.Symptome}]{\sphinxcrossref{\sphinxstyleliteralemphasis{\sphinxupquote{Symptome}}}}}) \textendash{} The symptom for which to create the text.

\sphinxlineitem{Returns}
\sphinxAtStartPar
The text to display containing the relevant information.

\sphinxlineitem{Return type}
\sphinxAtStartPar
str

\end{description}\end{quote}

\end{fulllineitems}

\index{update\_right\_panel() (general\_interface.InterfaceGenerale method)@\spxentry{update\_right\_panel()}\spxextra{general\_interface.InterfaceGenerale method}}

\begin{fulllineitems}
\phantomsection\label{\detokenize{general_interface:general_interface.InterfaceGenerale.update_right_panel}}
\pysigstartsignatures
\pysiglinewithargsret{\sphinxbfcode{\sphinxupquote{update\_right\_panel}}}{\sphinxparam{\DUrole{n}{attributs}\DUrole{o}{=}\DUrole{default_value}{{[}{]}}}\sphinxparamcomma \sphinxparam{\DUrole{n}{is\_start\_time}\DUrole{o}{=}\DUrole{default_value}{False}}}{}
\pysigstopsignatures
\sphinxAtStartPar
manage the display of symptom start/end times on the right\sphinxhyphen{}hand side and pop\sphinxhyphen{}up management to modify a symptom.
\begin{quote}\begin{description}
\sphinxlineitem{Parameters}
\sphinxAtStartPar
\sphinxstyleliteralstrong{\sphinxupquote{attributs}} (\sphinxstyleliteralemphasis{\sphinxupquote{list}}) \textendash{} symptom initialisation list | default = {[}{]}

\end{description}\end{quote}

\end{fulllineitems}


\end{fulllineitems}

\index{LecteurVideo (class in general\_interface)@\spxentry{LecteurVideo}\spxextra{class in general\_interface}}

\begin{fulllineitems}
\phantomsection\label{\detokenize{general_interface:general_interface.LecteurVideo}}
\pysigstartsignatures
\pysiglinewithargsret{\sphinxbfcode{\sphinxupquote{class\DUrole{w}{ }}}\sphinxcode{\sphinxupquote{general\_interface.}}\sphinxbfcode{\sphinxupquote{LecteurVideo}}}{\sphinxparam{\DUrole{n}{InterfaceGenerale}}}{}
\pysigstopsignatures
\sphinxAtStartPar
Bases: \sphinxcode{\sphinxupquote{object}}

\sphinxAtStartPar
Video player class that manages the different functionalities of the video.
\begin{description}
\sphinxlineitem{Functionalities supported :}\begin{itemize}
\item {} 
\sphinxAtStartPar
opening a video using the file explorer

\item {} 
\sphinxAtStartPar
opening a dark video \sphinxhyphen{} in case there is no video to actually open

\item {} 
\sphinxAtStartPar
sychronizing sound and video

\item {} 
\sphinxAtStartPar
buttons gestion

\end{itemize}

\end{description}

\sphinxAtStartPar
This class calls the InterfaceGenerale class
\index{interface\_generale (general\_interface.LecteurVideo attribute)@\spxentry{interface\_generale}\spxextra{general\_interface.LecteurVideo attribute}}

\begin{fulllineitems}
\phantomsection\label{\detokenize{general_interface:general_interface.LecteurVideo.interface_generale}}
\pysigstartsignatures
\pysigline{\sphinxbfcode{\sphinxupquote{interface\_generale}}}
\pysigstopsignatures
\sphinxAtStartPar
The main interface in which to open the video player.
\begin{quote}\begin{description}
\sphinxlineitem{Type}
\sphinxAtStartPar
{\hyperref[\detokenize{general_interface:general_interface.InterfaceGenerale}]{\sphinxcrossref{InterfaceGenerale}}}

\end{description}\end{quote}

\end{fulllineitems}

\index{video\_paused (general\_interface.LecteurVideo attribute)@\spxentry{video\_paused}\spxextra{general\_interface.LecteurVideo attribute}}

\begin{fulllineitems}
\phantomsection\label{\detokenize{general_interface:general_interface.LecteurVideo.video_paused}}
\pysigstartsignatures
\pysigline{\sphinxbfcode{\sphinxupquote{video\_paused}}}
\pysigstopsignatures
\sphinxAtStartPar
Indicates whether the video is currently paused.
\begin{quote}\begin{description}
\sphinxlineitem{Type}
\sphinxAtStartPar
bool

\end{description}\end{quote}

\end{fulllineitems}

\index{current\_frame\_time (general\_interface.LecteurVideo attribute)@\spxentry{current\_frame\_time}\spxextra{general\_interface.LecteurVideo attribute}}

\begin{fulllineitems}
\phantomsection\label{\detokenize{general_interface:general_interface.LecteurVideo.current_frame_time}}
\pysigstartsignatures
\pysigline{\sphinxbfcode{\sphinxupquote{current\_frame\_time}}}
\pysigstopsignatures
\sphinxAtStartPar
Current time of the video frame being displayed.
\begin{quote}\begin{description}
\sphinxlineitem{Type}
\sphinxAtStartPar
int

\end{description}\end{quote}

\end{fulllineitems}

\index{clock (general\_interface.LecteurVideo attribute)@\spxentry{clock}\spxextra{general\_interface.LecteurVideo attribute}}

\begin{fulllineitems}
\phantomsection\label{\detokenize{general_interface:general_interface.LecteurVideo.clock}}
\pysigstartsignatures
\pysigline{\sphinxbfcode{\sphinxupquote{clock}}}
\pysigstopsignatures
\sphinxAtStartPar
Pygame clock to manage time\sphinxhyphen{}related operations.
\begin{quote}\begin{description}
\sphinxlineitem{Type}
\sphinxAtStartPar
Clock

\end{description}\end{quote}

\end{fulllineitems}

\index{vitesse\_lecture (general\_interface.LecteurVideo attribute)@\spxentry{vitesse\_lecture}\spxextra{general\_interface.LecteurVideo attribute}}

\begin{fulllineitems}
\phantomsection\label{\detokenize{general_interface:general_interface.LecteurVideo.vitesse_lecture}}
\pysigstartsignatures
\pysigline{\sphinxbfcode{\sphinxupquote{vitesse\_lecture}}}
\pysigstopsignatures
\sphinxAtStartPar
Speed factor for video playback.
\begin{quote}\begin{description}
\sphinxlineitem{Type}
\sphinxAtStartPar
int

\end{description}\end{quote}

\end{fulllineitems}

\index{valeur (general\_interface.LecteurVideo attribute)@\spxentry{valeur}\spxextra{general\_interface.LecteurVideo attribute}}

\begin{fulllineitems}
\phantomsection\label{\detokenize{general_interface:general_interface.LecteurVideo.valeur}}
\pysigstartsignatures
\pysigline{\sphinxbfcode{\sphinxupquote{valeur}}}
\pysigstopsignatures
\sphinxAtStartPar
A value indicating something, not clearly specified in the docstring.
\begin{quote}\begin{description}
\sphinxlineitem{Type}
\sphinxAtStartPar
int

\end{description}\end{quote}

\end{fulllineitems}

\index{temps\_ecoule (general\_interface.LecteurVideo attribute)@\spxentry{temps\_ecoule}\spxextra{general\_interface.LecteurVideo attribute}}

\begin{fulllineitems}
\phantomsection\label{\detokenize{general_interface:general_interface.LecteurVideo.temps_ecoule}}
\pysigstartsignatures
\pysigline{\sphinxbfcode{\sphinxupquote{temps\_ecoule}}}
\pysigstopsignatures
\sphinxAtStartPar
Elapsed time of the video playback.
\begin{quote}\begin{description}
\sphinxlineitem{Type}
\sphinxAtStartPar
int

\end{description}\end{quote}

\end{fulllineitems}

\index{duree\_totale (general\_interface.LecteurVideo attribute)@\spxentry{duree\_totale}\spxextra{general\_interface.LecteurVideo attribute}}

\begin{fulllineitems}
\phantomsection\label{\detokenize{general_interface:general_interface.LecteurVideo.duree_totale}}
\pysigstartsignatures
\pysigline{\sphinxbfcode{\sphinxupquote{duree\_totale}}}
\pysigstopsignatures
\sphinxAtStartPar
Total duration of the loaded video.
\begin{quote}\begin{description}
\sphinxlineitem{Type}
\sphinxAtStartPar
int

\end{description}\end{quote}

\end{fulllineitems}

\index{afficher\_menu\_annotations() (general\_interface.LecteurVideo method)@\spxentry{afficher\_menu\_annotations()}\spxextra{general\_interface.LecteurVideo method}}

\begin{fulllineitems}
\phantomsection\label{\detokenize{general_interface:general_interface.LecteurVideo.afficher_menu_annotations}}
\pysigstartsignatures
\pysiglinewithargsret{\sphinxbfcode{\sphinxupquote{afficher\_menu\_annotations}}}{\sphinxparam{\DUrole{n}{event}}}{}
\pysigstopsignatures
\sphinxAtStartPar
Retrieve the coordinates by clicking on the video.

\sphinxAtStartPar
It currently displays a red cross.
But you can use this red cross to display the symtoms by hovering over it with the cursor.

\end{fulllineitems}

\index{afficher\_video() (general\_interface.LecteurVideo method)@\spxentry{afficher\_video()}\spxextra{general\_interface.LecteurVideo method}}

\begin{fulllineitems}
\phantomsection\label{\detokenize{general_interface:general_interface.LecteurVideo.afficher_video}}
\pysigstartsignatures
\pysiglinewithargsret{\sphinxbfcode{\sphinxupquote{afficher\_video}}}{}{}
\pysigstopsignatures
\sphinxAtStartPar
Allows to display and play the video.

\sphinxAtStartPar
Resizes the frames of the video to match thoses  of the middle frame
computes and displays the total and elapsed time of the video

\end{fulllineitems}

\index{afficher\_video\_frame() (general\_interface.LecteurVideo method)@\spxentry{afficher\_video\_frame()}\spxextra{general\_interface.LecteurVideo method}}

\begin{fulllineitems}
\phantomsection\label{\detokenize{general_interface:general_interface.LecteurVideo.afficher_video_frame}}
\pysigstartsignatures
\pysiglinewithargsret{\sphinxbfcode{\sphinxupquote{afficher\_video\_frame}}}{}{}
\pysigstopsignatures
\sphinxAtStartPar
Displays the current video frame without continuing the playback.

\sphinxAtStartPar
This method reads the current frame from the video, displays it on the canvas, and
stops the video from advancing. It also handles converting the frame to RGB color
space, resizing it to fit the canvas dimensions, and updating the canvas image.
\begin{quote}\begin{description}
\sphinxlineitem{Parameters}
\sphinxAtStartPar
\sphinxstyleliteralstrong{\sphinxupquote{None}} \textendash{} 

\sphinxlineitem{Returns}
\sphinxAtStartPar
None

\sphinxlineitem{Raises}
\sphinxAtStartPar
\sphinxstyleliteralstrong{\sphinxupquote{None}} \textendash{} 

\end{description}\end{quote}

\end{fulllineitems}

\index{ajouter\_plus\_rouge() (general\_interface.LecteurVideo method)@\spxentry{ajouter\_plus\_rouge()}\spxextra{general\_interface.LecteurVideo method}}

\begin{fulllineitems}
\phantomsection\label{\detokenize{general_interface:general_interface.LecteurVideo.ajouter_plus_rouge}}
\pysigstartsignatures
\pysiglinewithargsret{\sphinxbfcode{\sphinxupquote{ajouter\_plus\_rouge}}}{\sphinxparam{\DUrole{n}{canvas}}\sphinxparamcomma \sphinxparam{\DUrole{n}{x}}\sphinxparamcomma \sphinxparam{\DUrole{n}{y}}\sphinxparamcomma \sphinxparam{\DUrole{n}{taille}}}{}
\pysigstopsignatures
\sphinxAtStartPar
Displays the red cross for 1 second where one clicks on the video
\begin{quote}\begin{description}
\sphinxlineitem{Parameters}\begin{itemize}
\item {} 
\sphinxAtStartPar
\sphinxstyleliteralstrong{\sphinxupquote{canvas}} (\sphinxstyleliteralemphasis{\sphinxupquote{tk.Canvas}}) \textendash{} the canvas of the video

\item {} 
\sphinxAtStartPar
\sphinxstyleliteralstrong{\sphinxupquote{x}} (\sphinxstyleliteralemphasis{\sphinxupquote{int}}) \textendash{} x\sphinxhyphen{}coordinate of the cross

\item {} 
\sphinxAtStartPar
\sphinxstyleliteralstrong{\sphinxupquote{y}} (\sphinxstyleliteralemphasis{\sphinxupquote{int}}) \textendash{} y\sphinxhyphen{}coordinate of the cross

\item {} 
\sphinxAtStartPar
\sphinxstyleliteralstrong{\sphinxupquote{taille}} (\sphinxstyleliteralemphasis{\sphinxupquote{int}}) \textendash{} size of the cross

\end{itemize}

\end{description}\end{quote}

\end{fulllineitems}

\index{avance\_progress() (general\_interface.LecteurVideo method)@\spxentry{avance\_progress()}\spxextra{general\_interface.LecteurVideo method}}

\begin{fulllineitems}
\phantomsection\label{\detokenize{general_interface:general_interface.LecteurVideo.avance_progress}}
\pysigstartsignatures
\pysiglinewithargsret{\sphinxbfcode{\sphinxupquote{avance\_progress}}}{}{}
\pysigstopsignatures
\sphinxAtStartPar
Advances the video by a short duration and handles video and audio pausing correctly.

\sphinxAtStartPar
This method advances the video by 0.25 seconds (by default) and ensures that both video
and audio are paused if the video was originally playing.
\begin{quote}\begin{description}
\sphinxlineitem{Parameters}
\sphinxAtStartPar
\sphinxstyleliteralstrong{\sphinxupquote{None}} \textendash{} 

\sphinxlineitem{Returns}
\sphinxAtStartPar
None

\sphinxlineitem{Raises}
\sphinxAtStartPar
\sphinxstyleliteralstrong{\sphinxupquote{None}} \textendash{} 

\end{description}\end{quote}

\end{fulllineitems}

\index{avancer() (general\_interface.LecteurVideo method)@\spxentry{avancer()}\spxextra{general\_interface.LecteurVideo method}}

\begin{fulllineitems}
\phantomsection\label{\detokenize{general_interface:general_interface.LecteurVideo.avancer}}
\pysigstartsignatures
\pysiglinewithargsret{\sphinxbfcode{\sphinxupquote{avancer}}}{}{}
\pysigstopsignatures
\sphinxAtStartPar
Updates the position of the progress slider in the user interface.

\sphinxAtStartPar
This method sets the position of the progress slider to the elapsed time and
schedules itself to run again after a delay of 1 millisecond.
\begin{quote}\begin{description}
\sphinxlineitem{Parameters}
\sphinxAtStartPar
\sphinxstyleliteralstrong{\sphinxupquote{None}} \textendash{} 

\sphinxlineitem{Returns}
\sphinxAtStartPar
None

\sphinxlineitem{Raises}
\sphinxAtStartPar
\sphinxstyleliteralstrong{\sphinxupquote{None}} \textendash{} 

\end{description}\end{quote}

\end{fulllineitems}

\index{charger\_son\_video() (general\_interface.LecteurVideo method)@\spxentry{charger\_son\_video()}\spxextra{general\_interface.LecteurVideo method}}

\begin{fulllineitems}
\phantomsection\label{\detokenize{general_interface:general_interface.LecteurVideo.charger_son_video}}
\pysigstartsignatures
\pysiglinewithargsret{\sphinxbfcode{\sphinxupquote{charger\_son\_video}}}{\sphinxparam{\DUrole{n}{file\_path}}}{}
\pysigstopsignatures
\sphinxAtStartPar
Stops the previous audio, extracts and loads the new audio track from the selected video.

\sphinxAtStartPar
This method stops the previous audio playback, extracts the audio track from the
selected video file, saves it as a temporary WAV file, and loads it into the pygame
mixer for playback. It also cleans up the resources used during the process.
\begin{quote}\begin{description}
\sphinxlineitem{Parameters}
\sphinxAtStartPar
\sphinxstyleliteralstrong{\sphinxupquote{file\_path}} (\sphinxstyleliteralemphasis{\sphinxupquote{str}}) \textendash{} The path to the video file from which to extract the audio.

\sphinxlineitem{Returns}
\sphinxAtStartPar
None

\sphinxlineitem{Raises}
\sphinxAtStartPar
\sphinxstyleliteralstrong{\sphinxupquote{None}} \textendash{} 

\end{description}\end{quote}

\end{fulllineitems}

\index{configurer\_barre\_progression() (general\_interface.LecteurVideo method)@\spxentry{configurer\_barre\_progression()}\spxextra{general\_interface.LecteurVideo method}}

\begin{fulllineitems}
\phantomsection\label{\detokenize{general_interface:general_interface.LecteurVideo.configurer_barre_progression}}
\pysigstartsignatures
\pysiglinewithargsret{\sphinxbfcode{\sphinxupquote{configurer\_barre\_progression}}}{}{}
\pysigstopsignatures
\sphinxAtStartPar
Configures the progress bar’s range and binds events for slider interactions.

\sphinxAtStartPar
This method sets the total duration of the video as the range for the progress slider.
It also binds events for when the slider is pressed and released to handle dragging.
\begin{quote}\begin{description}
\sphinxlineitem{Parameters}
\sphinxAtStartPar
\sphinxstyleliteralstrong{\sphinxupquote{None}} \textendash{} 

\sphinxlineitem{Returns}
\sphinxAtStartPar
None

\sphinxlineitem{Raises}
\sphinxAtStartPar
\sphinxstyleliteralstrong{\sphinxupquote{None}} \textendash{} 

\end{description}\end{quote}

\end{fulllineitems}

\index{format\_duree() (general\_interface.LecteurVideo method)@\spxentry{format\_duree()}\spxextra{general\_interface.LecteurVideo method}}

\begin{fulllineitems}
\phantomsection\label{\detokenize{general_interface:general_interface.LecteurVideo.format_duree}}
\pysigstartsignatures
\pysiglinewithargsret{\sphinxbfcode{\sphinxupquote{format\_duree}}}{\sphinxparam{\DUrole{n}{seconds}}}{}
\pysigstopsignatures
\sphinxAtStartPar
Converts the time in seconds to a more readable format: minutes and seconds.

\sphinxAtStartPar
This method takes a duration in seconds and converts it to a string representation
in the format “HH:MM:SS”.
\begin{quote}\begin{description}
\sphinxlineitem{Parameters}
\sphinxAtStartPar
\sphinxstyleliteralstrong{\sphinxupquote{seconds}} (\sphinxstyleliteralemphasis{\sphinxupquote{int}}\sphinxstyleliteralemphasis{\sphinxupquote{ or }}\sphinxstyleliteralemphasis{\sphinxupquote{float}}) \textendash{} The time duration in seconds.

\sphinxlineitem{Returns}
\sphinxAtStartPar
The formatted time string in “HH:MM:SS” format.

\sphinxlineitem{Return type}
\sphinxAtStartPar
str

\sphinxlineitem{Raises}
\sphinxAtStartPar
\sphinxstyleliteralstrong{\sphinxupquote{None}} \textendash{} 

\end{description}\end{quote}
\subsubsection*{Example}

\begin{sphinxVerbatim}[commandchars=\\\{\}]
\PYG{g+gp}{\PYGZgt{}\PYGZgt{}\PYGZgt{} }\PYG{n}{player} \PYG{o}{=} \PYG{n}{LecteurVideo}\PYG{p}{(}\PYG{p}{)}
\PYG{g+gp}{\PYGZgt{}\PYGZgt{}\PYGZgt{} }\PYG{n}{formatted\PYGZus{}time} \PYG{o}{=} \PYG{n}{player}\PYG{o}{.}\PYG{n}{format\PYGZus{}duree}\PYG{p}{(}\PYG{l+m+mi}{3665}\PYG{p}{)}
\PYG{g+go}{\PYGZdq{}01:01:05\PYGZdq{}}
\end{sphinxVerbatim}

\end{fulllineitems}

\index{manual\_update\_progress() (general\_interface.LecteurVideo method)@\spxentry{manual\_update\_progress()}\spxextra{general\_interface.LecteurVideo method}}

\begin{fulllineitems}
\phantomsection\label{\detokenize{general_interface:general_interface.LecteurVideo.manual_update_progress}}
\pysigstartsignatures
\pysiglinewithargsret{\sphinxbfcode{\sphinxupquote{manual\_update\_progress}}}{\sphinxparam{\DUrole{n}{value}}}{}
\pysigstopsignatures
\sphinxAtStartPar
Manually updates the video position without starting the audio.

\sphinxAtStartPar
This method updates the elapsed time based on the given value, calculates the frame number
corresponding to the elapsed time, sets the video to that frame, and resumes audio playback
from the updated position if the video was playing.
\begin{quote}\begin{description}
\sphinxlineitem{Parameters}
\sphinxAtStartPar
\sphinxstyleliteralstrong{\sphinxupquote{value}} (\sphinxstyleliteralemphasis{\sphinxupquote{float}}) \textendash{} The new position value for the video in seconds.

\sphinxlineitem{Returns}
\sphinxAtStartPar
None

\sphinxlineitem{Raises}
\sphinxAtStartPar
\sphinxstyleliteralstrong{\sphinxupquote{None}} \textendash{} 

\end{description}\end{quote}

\end{fulllineitems}

\index{mettre\_a\_jour\_frame\_video() (general\_interface.LecteurVideo method)@\spxentry{mettre\_a\_jour\_frame\_video()}\spxextra{general\_interface.LecteurVideo method}}

\begin{fulllineitems}
\phantomsection\label{\detokenize{general_interface:general_interface.LecteurVideo.mettre_a_jour_frame_video}}
\pysigstartsignatures
\pysiglinewithargsret{\sphinxbfcode{\sphinxupquote{mettre\_a\_jour\_frame\_video}}}{\sphinxparam{\DUrole{n}{frame}}}{}
\pysigstopsignatures
\sphinxAtStartPar
Checks if each frame of the video is correctly processed, resized and displayed in the GUI

\end{fulllineitems}

\index{mettre\_a\_jour\_temps\_video() (general\_interface.LecteurVideo method)@\spxentry{mettre\_a\_jour\_temps\_video()}\spxextra{general\_interface.LecteurVideo method}}

\begin{fulllineitems}
\phantomsection\label{\detokenize{general_interface:general_interface.LecteurVideo.mettre_a_jour_temps_video}}
\pysigstartsignatures
\pysiglinewithargsret{\sphinxbfcode{\sphinxupquote{mettre\_a\_jour\_temps\_video}}}{}{}
\pysigstopsignatures
\sphinxAtStartPar
Ensures that the displayed time of the video is regularly updated and that the video plays smoothly.
It also ensures the sinchronization between the video advancement and the real time

\end{fulllineitems}

\index{on\_drag\_end() (general\_interface.LecteurVideo method)@\spxentry{on\_drag\_end()}\spxextra{general\_interface.LecteurVideo method}}

\begin{fulllineitems}
\phantomsection\label{\detokenize{general_interface:general_interface.LecteurVideo.on_drag_end}}
\pysigstartsignatures
\pysiglinewithargsret{\sphinxbfcode{\sphinxupquote{on\_drag\_end}}}{\sphinxparam{\DUrole{n}{event}}}{}
\pysigstopsignatures
\sphinxAtStartPar
Method called when the progress bar slider is released after dragging.

\sphinxAtStartPar
This method is triggered when the user releases the progress bar slider after dragging.
It updates the video position based on the slider’s position and resumes video playback.
\begin{quote}\begin{description}
\sphinxlineitem{Parameters}
\sphinxAtStartPar
\sphinxstyleliteralstrong{\sphinxupquote{event}} (\sphinxstyleliteralemphasis{\sphinxupquote{Event}}) \textendash{} The event object containing information about the event.

\sphinxlineitem{Returns}
\sphinxAtStartPar
None

\sphinxlineitem{Raises}
\sphinxAtStartPar
\sphinxstyleliteralstrong{\sphinxupquote{None}} \textendash{} 

\end{description}\end{quote}

\end{fulllineitems}

\index{on\_drag\_start() (general\_interface.LecteurVideo method)@\spxentry{on\_drag\_start()}\spxextra{general\_interface.LecteurVideo method}}

\begin{fulllineitems}
\phantomsection\label{\detokenize{general_interface:general_interface.LecteurVideo.on_drag_start}}
\pysigstartsignatures
\pysiglinewithargsret{\sphinxbfcode{\sphinxupquote{on\_drag\_start}}}{\sphinxparam{\DUrole{n}{event}}}{}
\pysigstopsignatures
\sphinxAtStartPar
Method called when the progress bar slider is clicked and dragged.

\sphinxAtStartPar
This method is triggered when the user starts dragging the progress bar slider.
It pauses the video playback.
\begin{quote}\begin{description}
\sphinxlineitem{Parameters}
\sphinxAtStartPar
\sphinxstyleliteralstrong{\sphinxupquote{event}} (\sphinxstyleliteralemphasis{\sphinxupquote{Event}}) \textendash{} The event object containing information about the event.

\sphinxlineitem{Returns}
\sphinxAtStartPar
None

\sphinxlineitem{Raises}
\sphinxAtStartPar
\sphinxstyleliteralstrong{\sphinxupquote{None}} \textendash{} 

\end{description}\end{quote}

\end{fulllineitems}

\index{ouvrir\_video() (general\_interface.LecteurVideo method)@\spxentry{ouvrir\_video()}\spxextra{general\_interface.LecteurVideo method}}

\begin{fulllineitems}
\phantomsection\label{\detokenize{general_interface:general_interface.LecteurVideo.ouvrir_video}}
\pysigstartsignatures
\pysiglinewithargsret{\sphinxbfcode{\sphinxupquote{ouvrir\_video}}}{}{}
\pysigstopsignatures
\sphinxAtStartPar
Opens a video with the file explorer.
Once chosen, it plays the video in the middle\_frame

\end{fulllineitems}

\index{ouvrir\_video\_noire() (general\_interface.LecteurVideo method)@\spxentry{ouvrir\_video\_noire()}\spxextra{general\_interface.LecteurVideo method}}

\begin{fulllineitems}
\phantomsection\label{\detokenize{general_interface:general_interface.LecteurVideo.ouvrir_video_noire}}
\pysigstartsignatures
\pysiglinewithargsret{\sphinxbfcode{\sphinxupquote{ouvrir\_video\_noire}}}{}{}
\pysigstopsignatures
\sphinxAtStartPar
Opens the black video.
The black video must be in the same folder as the code : “video\_noire.mp4”

\end{fulllineitems}

\index{pause\_lecture() (general\_interface.LecteurVideo method)@\spxentry{pause\_lecture()}\spxextra{general\_interface.LecteurVideo method}}

\begin{fulllineitems}
\phantomsection\label{\detokenize{general_interface:general_interface.LecteurVideo.pause_lecture}}
\pysigstartsignatures
\pysiglinewithargsret{\sphinxbfcode{\sphinxupquote{pause\_lecture}}}{}{}
\pysigstopsignatures
\sphinxAtStartPar
Manage the play/pause state of the video

\end{fulllineitems}

\index{preparer\_mixer() (general\_interface.LecteurVideo method)@\spxentry{preparer\_mixer()}\spxextra{general\_interface.LecteurVideo method}}

\begin{fulllineitems}
\phantomsection\label{\detokenize{general_interface:general_interface.LecteurVideo.preparer_mixer}}
\pysigstartsignatures
\pysiglinewithargsret{\sphinxbfcode{\sphinxupquote{preparer\_mixer}}}{\sphinxparam{\DUrole{n}{file\_path}}}{}
\pysigstopsignatures
\sphinxAtStartPar
Initializes pygame mixer with audio properties extracted from a video file using moviepy.
\begin{quote}\begin{description}
\sphinxlineitem{Parameters}
\sphinxAtStartPar
\sphinxstyleliteralstrong{\sphinxupquote{file\_path}} (\sphinxstyleliteralemphasis{\sphinxupquote{str}}) \textendash{} The path to the video file.

\sphinxlineitem{Returns}
\sphinxAtStartPar
None

\sphinxlineitem{Raises}
\sphinxAtStartPar
\sphinxstyleliteralstrong{\sphinxupquote{None}} \textendash{} 

\end{description}\end{quote}

\end{fulllineitems}

\index{preparer\_son\_video() (general\_interface.LecteurVideo method)@\spxentry{preparer\_son\_video()}\spxextra{general\_interface.LecteurVideo method}}

\begin{fulllineitems}
\phantomsection\label{\detokenize{general_interface:general_interface.LecteurVideo.preparer_son_video}}
\pysigstartsignatures
\pysiglinewithargsret{\sphinxbfcode{\sphinxupquote{preparer\_son\_video}}}{\sphinxparam{\DUrole{n}{file\_path}}}{}
\pysigstopsignatures
\sphinxAtStartPar
Prepares the mixer for each video and plays its audio.
\begin{quote}\begin{description}
\sphinxlineitem{Parameters}
\sphinxAtStartPar
\sphinxstyleliteralstrong{\sphinxupquote{file\_path}} (\sphinxstyleliteralemphasis{\sphinxupquote{str}}) \textendash{} The path to the video file.

\sphinxlineitem{Returns}
\sphinxAtStartPar
None

\sphinxlineitem{Raises}
\sphinxAtStartPar
\sphinxstyleliteralstrong{\sphinxupquote{None}} \textendash{} 

\end{description}\end{quote}

\end{fulllineitems}

\index{recule\_progress() (general\_interface.LecteurVideo method)@\spxentry{recule\_progress()}\spxextra{general\_interface.LecteurVideo method}}

\begin{fulllineitems}
\phantomsection\label{\detokenize{general_interface:general_interface.LecteurVideo.recule_progress}}
\pysigstartsignatures
\pysiglinewithargsret{\sphinxbfcode{\sphinxupquote{recule\_progress}}}{}{}
\pysigstopsignatures
\sphinxAtStartPar
Rewinds the video by a short duration and handles video and audio pausing correctly.

\sphinxAtStartPar
This method rewinds the video by 0.25 seconds (by default) and ensures that both video
and audio are paused if the video was originally playing.
\begin{quote}\begin{description}
\sphinxlineitem{Parameters}
\sphinxAtStartPar
\sphinxstyleliteralstrong{\sphinxupquote{None}} \textendash{} 

\sphinxlineitem{Returns}
\sphinxAtStartPar
None

\sphinxlineitem{Raises}
\sphinxAtStartPar
\sphinxstyleliteralstrong{\sphinxupquote{None}} \textendash{} 

\end{description}\end{quote}

\end{fulllineitems}

\index{resume\_lecture() (general\_interface.LecteurVideo method)@\spxentry{resume\_lecture()}\spxextra{general\_interface.LecteurVideo method}}

\begin{fulllineitems}
\phantomsection\label{\detokenize{general_interface:general_interface.LecteurVideo.resume_lecture}}
\pysigstartsignatures
\pysiglinewithargsret{\sphinxbfcode{\sphinxupquote{resume\_lecture}}}{}{}
\pysigstopsignatures
\sphinxAtStartPar
Resumes video and audio playback.

\sphinxAtStartPar
This method resumes the paused video and audio playback, updates the play/pause button text,
unpause the audio mixer, displays the video, and updates the video time.
\begin{quote}\begin{description}
\sphinxlineitem{Parameters}
\sphinxAtStartPar
\sphinxstyleliteralstrong{\sphinxupquote{None}} \textendash{} 

\sphinxlineitem{Returns}
\sphinxAtStartPar
None

\sphinxlineitem{Raises}
\sphinxAtStartPar
\sphinxstyleliteralstrong{\sphinxupquote{None}} \textendash{} 

\end{description}\end{quote}

\end{fulllineitems}

\index{revoir\_video() (general\_interface.LecteurVideo method)@\spxentry{revoir\_video()}\spxextra{general\_interface.LecteurVideo method}}

\begin{fulllineitems}
\phantomsection\label{\detokenize{general_interface:general_interface.LecteurVideo.revoir_video}}
\pysigstartsignatures
\pysiglinewithargsret{\sphinxbfcode{\sphinxupquote{revoir\_video}}}{}{}
\pysigstopsignatures
\sphinxAtStartPar
Resets the video and audio to the beginning and resumes video playback.

\sphinxAtStartPar
This method sets the video frame position and audio playback to the beginning,
resumes video playback if it was paused, and updates the play/pause button text.
\begin{quote}\begin{description}
\sphinxlineitem{Parameters}
\sphinxAtStartPar
\sphinxstyleliteralstrong{\sphinxupquote{None}} \textendash{} 

\sphinxlineitem{Returns}
\sphinxAtStartPar
None

\sphinxlineitem{Raises}
\sphinxAtStartPar
\sphinxstyleliteralstrong{\sphinxupquote{None}} \textendash{} 

\end{description}\end{quote}

\end{fulllineitems}


\end{fulllineitems}

\index{Menu\_symptomes (class in general\_interface)@\spxentry{Menu\_symptomes}\spxextra{class in general\_interface}}

\begin{fulllineitems}
\phantomsection\label{\detokenize{general_interface:general_interface.Menu_symptomes}}
\pysigstartsignatures
\pysiglinewithargsret{\sphinxbfcode{\sphinxupquote{class\DUrole{w}{ }}}\sphinxcode{\sphinxupquote{general\_interface.}}\sphinxbfcode{\sphinxupquote{Menu\_symptomes}}}{\sphinxparam{\DUrole{n}{master}}\sphinxparamcomma \sphinxparam{\DUrole{n}{interface\_generale}}\sphinxparamcomma \sphinxparam{\DUrole{n}{couleur}}\sphinxparamcomma \sphinxparam{\DUrole{n}{bordure}}\sphinxparamcomma \sphinxparam{\DUrole{n}{largeur}}}{}
\pysigstopsignatures
\sphinxAtStartPar
Bases: \sphinxcode{\sphinxupquote{CTkFrame}}

\sphinxAtStartPar
Class used to instantiate drop\sphinxhyphen{}down menus displaying symptoms in a frame on the left of the interface.
\index{create\_dropdown\_menus() (general\_interface.Menu\_symptomes method)@\spxentry{create\_dropdown\_menus()}\spxextra{general\_interface.Menu\_symptomes method}}

\begin{fulllineitems}
\phantomsection\label{\detokenize{general_interface:general_interface.Menu_symptomes.create_dropdown_menus}}
\pysigstartsignatures
\pysiglinewithargsret{\sphinxbfcode{\sphinxupquote{create\_dropdown\_menus}}}{\sphinxparam{\DUrole{n}{largeur}}}{}
\pysigstopsignatures
\sphinxAtStartPar
Creates a drop\sphinxhyphen{}down menu containing the various symptoms classified according to whether they are objective or subjective.
\begin{quote}\begin{description}
\sphinxlineitem{Parameters}
\sphinxAtStartPar
\sphinxstyleliteralstrong{\sphinxupquote{largeur}} (\sphinxstyleliteralemphasis{\sphinxupquote{int}}) \textendash{} width of the frame of the menus

\end{description}\end{quote}

\end{fulllineitems}

\index{on\_select() (general\_interface.Menu\_symptomes method)@\spxentry{on\_select()}\spxextra{general\_interface.Menu\_symptomes method}}

\begin{fulllineitems}
\phantomsection\label{\detokenize{general_interface:general_interface.Menu_symptomes.on_select}}
\pysigstartsignatures
\pysiglinewithargsret{\sphinxbfcode{\sphinxupquote{on\_select}}}{\sphinxparam{\DUrole{n}{selection}}}{}
\pysigstopsignatures
\sphinxAtStartPar
Symptoms selector

\sphinxAtStartPar
Sets the right attributes depending on the menu selection to update the symptom list
Recovers data linked to the video to obtain the current time
the selection is expected to be on the form
designation \textgreater{} description \textgreater{} sub descrition\textgreater{} topography \textgreater{} lateralization
\begin{quote}\begin{description}
\sphinxlineitem{Parameters}
\sphinxAtStartPar
\sphinxstyleliteralstrong{\sphinxupquote{selection}} (\sphinxstyleliteralemphasis{\sphinxupquote{path}}) \textendash{} selected symptom

\end{description}\end{quote}

\end{fulllineitems}

\index{on\_select\_bar() (general\_interface.Menu\_symptomes method)@\spxentry{on\_select\_bar()}\spxextra{general\_interface.Menu\_symptomes method}}

\begin{fulllineitems}
\phantomsection\label{\detokenize{general_interface:general_interface.Menu_symptomes.on_select_bar}}
\pysigstartsignatures
\pysiglinewithargsret{\sphinxbfcode{\sphinxupquote{on\_select\_bar}}}{\sphinxparam{\DUrole{n}{selection}}}{}
\pysigstopsignatures
\end{fulllineitems}


\end{fulllineitems}


\sphinxstepscope


\chapter{frise package}
\label{\detokenize{frise:frise-package}}\label{\detokenize{frise::doc}}

\section{Submodules}
\label{\detokenize{frise:submodules}}

\section{frise.ecriture\_fichier module}
\label{\detokenize{frise:module-frise.ecriture_fichier}}\label{\detokenize{frise:frise-ecriture-fichier-module}}\index{module@\spxentry{module}!frise.ecriture\_fichier@\spxentry{frise.ecriture\_fichier}}\index{frise.ecriture\_fichier@\spxentry{frise.ecriture\_fichier}!module@\spxentry{module}}
\sphinxAtStartPar
This module provides functions for writing symptom data, symptom lists, and metadata to text files.
\begin{description}
\sphinxlineitem{Functions:}
\sphinxAtStartPar
EcrireSymptome: Writes a single symptom to a file.
EcrireListeSymptome: Writes a list of symptoms to a file.
EcrireMetaData: Writes metadata to a file.
format: Formats text data by replacing specified characters.
ecrire\_rapport: Writes a report with symptom details.

\end{description}
\index{EcrireListeSymptome() (in module frise.ecriture\_fichier)@\spxentry{EcrireListeSymptome()}\spxextra{in module frise.ecriture\_fichier}}

\begin{fulllineitems}
\phantomsection\label{\detokenize{frise:frise.ecriture_fichier.EcrireListeSymptome}}
\pysigstartsignatures
\pysiglinewithargsret{\sphinxcode{\sphinxupquote{frise.ecriture\_fichier.}}\sphinxbfcode{\sphinxupquote{EcrireListeSymptome}}}{\sphinxparam{\DUrole{n}{listeSymptome}}\sphinxparamcomma \sphinxparam{\DUrole{n}{nomfichier}}}{}
\pysigstopsignatures
\sphinxAtStartPar
Writes a list of symptoms to a file.
\begin{quote}\begin{description}
\sphinxlineitem{Parameters}\begin{itemize}
\item {} 
\sphinxAtStartPar
\sphinxstyleliteralstrong{\sphinxupquote{listeSymptome}} (\sphinxstyleliteralemphasis{\sphinxupquote{list}}) \textendash{} List of symptom objects to be written to the file.

\item {} 
\sphinxAtStartPar
\sphinxstyleliteralstrong{\sphinxupquote{nomfichier}} (\sphinxstyleliteralemphasis{\sphinxupquote{str}}) \textendash{} The path of the file to write to.

\end{itemize}

\sphinxlineitem{Returns}
\sphinxAtStartPar
None

\end{description}\end{quote}

\end{fulllineitems}

\index{EcrireMetaData() (in module frise.ecriture\_fichier)@\spxentry{EcrireMetaData()}\spxextra{in module frise.ecriture\_fichier}}

\begin{fulllineitems}
\phantomsection\label{\detokenize{frise:frise.ecriture_fichier.EcrireMetaData}}
\pysigstartsignatures
\pysiglinewithargsret{\sphinxcode{\sphinxupquote{frise.ecriture\_fichier.}}\sphinxbfcode{\sphinxupquote{EcrireMetaData}}}{\sphinxparam{\DUrole{n}{Meta}}\sphinxparamcomma \sphinxparam{\DUrole{n}{nomfichier}}}{}
\pysigstopsignatures
\sphinxAtStartPar
Writes metadata to a file.
\begin{quote}\begin{description}
\sphinxlineitem{Parameters}\begin{itemize}
\item {} 
\sphinxAtStartPar
\sphinxstyleliteralstrong{\sphinxupquote{Meta}} (\sphinxcode{\sphinxupquote{Matadata}}) \textendash{} metadata information.

\item {} 
\sphinxAtStartPar
\sphinxstyleliteralstrong{\sphinxupquote{nomfichier}} (\sphinxstyleliteralemphasis{\sphinxupquote{str}}) \textendash{} The path of the file to write to.

\end{itemize}

\sphinxlineitem{Returns}
\sphinxAtStartPar
None

\end{description}\end{quote}

\end{fulllineitems}

\index{EcrireSymptome() (in module frise.ecriture\_fichier)@\spxentry{EcrireSymptome()}\spxextra{in module frise.ecriture\_fichier}}

\begin{fulllineitems}
\phantomsection\label{\detokenize{frise:frise.ecriture_fichier.EcrireSymptome}}
\pysigstartsignatures
\pysiglinewithargsret{\sphinxcode{\sphinxupquote{frise.ecriture\_fichier.}}\sphinxbfcode{\sphinxupquote{EcrireSymptome}}}{\sphinxparam{\DUrole{n}{symptome}}\sphinxparamcomma \sphinxparam{\DUrole{n}{nomfichier}}}{}
\pysigstopsignatures
\sphinxAtStartPar
Writes a single symptom to a file.
\begin{quote}\begin{description}
\sphinxlineitem{Parameters}\begin{itemize}
\item {} 
\sphinxAtStartPar
\sphinxstyleliteralstrong{\sphinxupquote{symptome}} (\sphinxcode{\sphinxupquote{Symptome}}) \textendash{} The symptom object to be written to the file.

\item {} 
\sphinxAtStartPar
\sphinxstyleliteralstrong{\sphinxupquote{nomfichier}} (\sphinxstyleliteralemphasis{\sphinxupquote{str}}) \textendash{} The path of the file to write to.

\end{itemize}

\sphinxlineitem{Returns}
\sphinxAtStartPar
None

\end{description}\end{quote}

\end{fulllineitems}

\index{compute\_duration() (in module frise.ecriture\_fichier)@\spxentry{compute\_duration()}\spxextra{in module frise.ecriture\_fichier}}

\begin{fulllineitems}
\phantomsection\label{\detokenize{frise:frise.ecriture_fichier.compute_duration}}
\pysigstartsignatures
\pysiglinewithargsret{\sphinxcode{\sphinxupquote{frise.ecriture\_fichier.}}\sphinxbfcode{\sphinxupquote{compute\_duration}}}{\sphinxparam{\DUrole{n}{deb}}\sphinxparamcomma \sphinxparam{\DUrole{n}{fin}}}{}
\pysigstopsignatures
\sphinxAtStartPar
Compute the duration between two time strings in seconds.
\begin{quote}\begin{description}
\sphinxlineitem{Parameters}\begin{itemize}
\item {} 
\sphinxAtStartPar
\sphinxstyleliteralstrong{\sphinxupquote{deb}} (\sphinxstyleliteralemphasis{\sphinxupquote{str}}) \textendash{} Start time string in the format “hh:mm:ss”.

\item {} 
\sphinxAtStartPar
\sphinxstyleliteralstrong{\sphinxupquote{fin}} (\sphinxstyleliteralemphasis{\sphinxupquote{str}}) \textendash{} End time string in the format “hh:mm:ss”.

\end{itemize}

\sphinxlineitem{Returns}
\sphinxAtStartPar
Total duration between the start and end time in seconds.

\sphinxlineitem{Return type}
\sphinxAtStartPar
int

\end{description}\end{quote}

\end{fulllineitems}

\index{ecrire\_rapport() (in module frise.ecriture\_fichier)@\spxentry{ecrire\_rapport()}\spxextra{in module frise.ecriture\_fichier}}

\begin{fulllineitems}
\phantomsection\label{\detokenize{frise:frise.ecriture_fichier.ecrire_rapport}}
\pysigstartsignatures
\pysiglinewithargsret{\sphinxcode{\sphinxupquote{frise.ecriture\_fichier.}}\sphinxbfcode{\sphinxupquote{ecrire\_rapport}}}{\sphinxparam{\DUrole{n}{Symptom\_list}}\sphinxparamcomma \sphinxparam{\DUrole{n}{filename}}}{}
\pysigstopsignatures
\sphinxAtStartPar
Writes a report with symptom details to a file.
\begin{quote}\begin{description}
\sphinxlineitem{Parameters}\begin{itemize}
\item {} 
\sphinxAtStartPar
\sphinxstyleliteralstrong{\sphinxupquote{Symptom\_list}} (\sphinxcode{\sphinxupquote{list}} of \sphinxcode{\sphinxupquote{Symptome}}) \textendash{} List of symptom objects.

\item {} 
\sphinxAtStartPar
\sphinxstyleliteralstrong{\sphinxupquote{filename}} (\sphinxstyleliteralemphasis{\sphinxupquote{str}}) \textendash{} The path of the file to write to.

\end{itemize}

\end{description}\end{quote}

\end{fulllineitems}

\index{format() (in module frise.ecriture\_fichier)@\spxentry{format()}\spxextra{in module frise.ecriture\_fichier}}

\begin{fulllineitems}
\phantomsection\label{\detokenize{frise:frise.ecriture_fichier.format}}
\pysigstartsignatures
\pysiglinewithargsret{\sphinxcode{\sphinxupquote{frise.ecriture\_fichier.}}\sphinxbfcode{\sphinxupquote{format}}}{\sphinxparam{\DUrole{n}{data}}\sphinxparamcomma \sphinxparam{\DUrole{n}{caracteres}}}{}
\pysigstopsignatures
\sphinxAtStartPar
Formats text data by replacing specified characters.
\begin{quote}\begin{description}
\sphinxlineitem{Parameters}\begin{itemize}
\item {} 
\sphinxAtStartPar
\sphinxstyleliteralstrong{\sphinxupquote{data}} (\sphinxcode{\sphinxupquote{list}} of \sphinxcode{\sphinxupquote{str}}) \textendash{} List of text data to be formatted.

\item {} 
\sphinxAtStartPar
\sphinxstyleliteralstrong{\sphinxupquote{caracteres}} (\sphinxstyleliteralemphasis{\sphinxupquote{List}}) \textendash{} List containing character replacement pairs.

\end{itemize}

\sphinxlineitem{Returns}
\sphinxAtStartPar
Formatted text data.

\sphinxlineitem{Return type}
\sphinxAtStartPar
\sphinxcode{\sphinxupquote{list}} of \sphinxcode{\sphinxupquote{str}}

\end{description}\end{quote}

\end{fulllineitems}



\section{frise.fonctions\_frise module}
\label{\detokenize{frise:module-frise.fonctions_frise}}\label{\detokenize{frise:frise-fonctions-frise-module}}\index{module@\spxentry{module}!frise.fonctions\_frise@\spxentry{frise.fonctions\_frise}}\index{frise.fonctions\_frise@\spxentry{frise.fonctions\_frise}!module@\spxentry{module}}
\sphinxAtStartPar
This module provides functions for visualizing a chronological timeline of symptoms.
\begin{description}
\sphinxlineitem{Functions:}
\sphinxAtStartPar
chercherElt(list): Searches for a missing element in a list of integers.
chevauchement(liste, symp, current\_index, levels): Manages the visual overlapping of symptoms.
on\_text\_click(event): Displays the annotation when the text is clicked.
afficher\_frise(liste): Displays the chronological timeline of symptoms.

\end{description}
\index{afficher\_frise() (in module frise.fonctions\_frise)@\spxentry{afficher\_frise()}\spxextra{in module frise.fonctions\_frise}}

\begin{fulllineitems}
\phantomsection\label{\detokenize{frise:frise.fonctions_frise.afficher_frise}}
\pysigstartsignatures
\pysiglinewithargsret{\sphinxcode{\sphinxupquote{frise.fonctions\_frise.}}\sphinxbfcode{\sphinxupquote{afficher\_frise}}}{\sphinxparam{\DUrole{n}{liste}}}{}
\pysigstopsignatures
\sphinxAtStartPar
Displays the chronological timeline of symptoms.
\begin{quote}\begin{description}
\sphinxlineitem{Parameters}
\sphinxAtStartPar
\sphinxstyleliteralstrong{\sphinxupquote{(}} (\sphinxstyleliteralemphasis{\sphinxupquote{liste}}) \textendash{} obj:list of :obj:list): List of symptoms where each element is a list {[}Name, start, end, Lateralization, seg corporel, orientation, additional attribute, Comment, tdeb\_str, tfin\_str{]}.

\end{description}\end{quote}

\end{fulllineitems}

\index{chercherElt() (in module frise.fonctions\_frise)@\spxentry{chercherElt()}\spxextra{in module frise.fonctions\_frise}}

\begin{fulllineitems}
\phantomsection\label{\detokenize{frise:frise.fonctions_frise.chercherElt}}
\pysigstartsignatures
\pysiglinewithargsret{\sphinxcode{\sphinxupquote{frise.fonctions\_frise.}}\sphinxbfcode{\sphinxupquote{chercherElt}}}{\sphinxparam{\DUrole{n}{list}}}{}
\pysigstopsignatures
\sphinxAtStartPar
Searches for a missing element in a list of integers.
\begin{quote}\begin{description}
\sphinxlineitem{Parameters}
\sphinxAtStartPar
\sphinxstyleliteralstrong{\sphinxupquote{list}} (\sphinxstyleliteralemphasis{\sphinxupquote{list}}) \textendash{} A list of integers.

\sphinxlineitem{Returns}
\sphinxAtStartPar
The missing element in the list.

\sphinxlineitem{Return type}
\sphinxAtStartPar
int

\end{description}\end{quote}

\end{fulllineitems}

\index{chevauchement() (in module frise.fonctions\_frise)@\spxentry{chevauchement()}\spxextra{in module frise.fonctions\_frise}}

\begin{fulllineitems}
\phantomsection\label{\detokenize{frise:frise.fonctions_frise.chevauchement}}
\pysigstartsignatures
\pysiglinewithargsret{\sphinxcode{\sphinxupquote{frise.fonctions\_frise.}}\sphinxbfcode{\sphinxupquote{chevauchement}}}{\sphinxparam{\DUrole{n}{liste}}\sphinxparamcomma \sphinxparam{\DUrole{n}{symp}}\sphinxparamcomma \sphinxparam{\DUrole{n}{current\_index}}\sphinxparamcomma \sphinxparam{\DUrole{n}{levels}}}{}
\pysigstopsignatures
\sphinxAtStartPar
Manages the visual overlapping of symptoms.
\begin{quote}\begin{description}
\sphinxlineitem{Parameters}\begin{itemize}
\item {} 
\sphinxAtStartPar
\sphinxstyleliteralstrong{\sphinxupquote{liste}} (\sphinxstyleliteralemphasis{\sphinxupquote{list}}) \textendash{} List of symptoms with start and end times.

\item {} 
\sphinxAtStartPar
\sphinxstyleliteralstrong{\sphinxupquote{symp}} (\sphinxcode{\sphinxupquote{Symptome}}) \textendash{} Current symptom.

\item {} 
\sphinxAtStartPar
\sphinxstyleliteralstrong{\sphinxupquote{current\_index}} (\sphinxstyleliteralemphasis{\sphinxupquote{int}}) \textendash{} Index of the current symptom in the list.

\item {} 
\sphinxAtStartPar
\sphinxstyleliteralstrong{\sphinxupquote{levels}} (\sphinxstyleliteralemphasis{\sphinxupquote{list}}) \textendash{} List of levels.

\end{itemize}

\sphinxlineitem{Returns}
\sphinxAtStartPar
The y\sphinxhyphen{}level where to display the rectangle.

\sphinxlineitem{Return type}
\sphinxAtStartPar
int

\end{description}\end{quote}

\end{fulllineitems}

\index{on\_text\_click() (in module frise.fonctions\_frise)@\spxentry{on\_text\_click()}\spxextra{in module frise.fonctions\_frise}}

\begin{fulllineitems}
\phantomsection\label{\detokenize{frise:frise.fonctions_frise.on_text_click}}
\pysigstartsignatures
\pysiglinewithargsret{\sphinxcode{\sphinxupquote{frise.fonctions\_frise.}}\sphinxbfcode{\sphinxupquote{on\_text\_click}}}{\sphinxparam{\DUrole{n}{event}}}{}
\pysigstopsignatures
\sphinxAtStartPar
Displays the annotation when the text is clicked.
\begin{quote}\begin{description}
\sphinxlineitem{Parameters}
\sphinxAtStartPar
\sphinxstyleliteralstrong{\sphinxupquote{event}} (\sphinxstyleliteralemphasis{\sphinxupquote{matplotlib.backend\_bases.MouseEvent}}) \textendash{} The mouse event.

\end{description}\end{quote}

\end{fulllineitems}



\section{frise.save module}
\label{\detokenize{frise:module-frise.save}}\label{\detokenize{frise:frise-save-module}}\index{module@\spxentry{module}!frise.save@\spxentry{frise.save}}\index{frise.save@\spxentry{frise.save}!module@\spxentry{module}}
\sphinxAtStartPar
This file contains basic classes and functions for saving ‘.txt’ files and outputting timelines.
\begin{description}
\sphinxlineitem{Classes:}
\sphinxAtStartPar
save: Class dedicated to saving files.
MetaData\_WD: Toplevel window for entering metadata.

\sphinxlineitem{Functions:}
\sphinxAtStartPar
None

\end{description}
\index{MetaData\_WD (class in frise.save)@\spxentry{MetaData\_WD}\spxextra{class in frise.save}}

\begin{fulllineitems}
\phantomsection\label{\detokenize{frise:frise.save.MetaData_WD}}
\pysigstartsignatures
\pysiglinewithargsret{\sphinxbfcode{\sphinxupquote{class\DUrole{w}{ }}}\sphinxcode{\sphinxupquote{frise.save.}}\sphinxbfcode{\sphinxupquote{MetaData\_WD}}}{\sphinxparam{\DUrole{n}{filename}}}{}
\pysigstopsignatures
\sphinxAtStartPar
Bases: \sphinxcode{\sphinxupquote{CTkToplevel}}

\sphinxAtStartPar
Toplevel window for entering metadata.
\index{liste (frise.save.MetaData\_WD attribute)@\spxentry{liste}\spxextra{frise.save.MetaData\_WD attribute}}

\begin{fulllineitems}
\phantomsection\label{\detokenize{frise:frise.save.MetaData_WD.liste}}
\pysigstartsignatures
\pysigline{\sphinxbfcode{\sphinxupquote{liste}}}
\pysigstopsignatures
\sphinxAtStartPar
List of symptoms to be saved.
\begin{quote}\begin{description}
\sphinxlineitem{Type}
\sphinxAtStartPar
list

\end{description}\end{quote}

\end{fulllineitems}

\index{filename (frise.save.MetaData\_WD attribute)@\spxentry{filename}\spxextra{frise.save.MetaData\_WD attribute}}

\begin{fulllineitems}
\phantomsection\label{\detokenize{frise:frise.save.MetaData_WD.filename}}
\pysigstartsignatures
\pysigline{\sphinxbfcode{\sphinxupquote{filename}}}
\pysigstopsignatures
\sphinxAtStartPar
Path of the file to write to.
\begin{quote}\begin{description}
\sphinxlineitem{Type}
\sphinxAtStartPar
string

\end{description}\end{quote}

\end{fulllineitems}

\index{get\_metadata() (frise.save.MetaData\_WD method)@\spxentry{get\_metadata()}\spxextra{frise.save.MetaData\_WD method}}

\begin{fulllineitems}
\phantomsection\label{\detokenize{frise:frise.save.MetaData_WD.get_metadata}}
\pysigstartsignatures
\pysiglinewithargsret{\sphinxbfcode{\sphinxupquote{get\_metadata}}}{\sphinxparam{\DUrole{n}{event}}}{}
\pysigstopsignatures
\sphinxAtStartPar
Write a list of metadata in the form {[}real time, patient, practitioner{]}.
\begin{quote}\begin{description}
\sphinxlineitem{Parameters}
\sphinxAtStartPar
\sphinxstyleliteralstrong{\sphinxupquote{event}} (\sphinxstyleliteralemphasis{\sphinxupquote{any}}) \textendash{} Corresponds to writing in the text boxes.

\end{description}\end{quote}

\end{fulllineitems}


\end{fulllineitems}

\index{save (class in frise.save)@\spxentry{save}\spxextra{class in frise.save}}

\begin{fulllineitems}
\phantomsection\label{\detokenize{frise:frise.save.save}}
\pysigstartsignatures
\pysiglinewithargsret{\sphinxbfcode{\sphinxupquote{class\DUrole{w}{ }}}\sphinxcode{\sphinxupquote{frise.save.}}\sphinxbfcode{\sphinxupquote{save}}}{\sphinxparam{\DUrole{n}{Liste\_symptomes}\DUrole{o}{=}\DUrole{default_value}{{[}{]}}}}{}
\pysigstopsignatures
\sphinxAtStartPar
Bases: \sphinxcode{\sphinxupquote{object}}

\sphinxAtStartPar
Class dedicated to saving files.
\index{symptomes (frise.save.save attribute)@\spxentry{symptomes}\spxextra{frise.save.save attribute}}

\begin{fulllineitems}
\phantomsection\label{\detokenize{frise:frise.save.save.symptomes}}
\pysigstartsignatures
\pysigline{\sphinxbfcode{\sphinxupquote{symptomes}}}
\pysigstopsignatures
\sphinxAtStartPar
List of symptoms to be saved
\begin{quote}\begin{description}
\sphinxlineitem{Type}
\sphinxAtStartPar
list

\end{description}\end{quote}

\end{fulllineitems}

\index{save() (frise.save.save method)@\spxentry{save()}\spxextra{frise.save.save method}}

\begin{fulllineitems}
\phantomsection\label{\detokenize{frise:frise.save.save.save}}
\pysigstartsignatures
\pysiglinewithargsret{\sphinxbfcode{\sphinxupquote{save}}}{}{}
\pysigstopsignatures
\sphinxAtStartPar
Saves the symptoms to a file for reloading.
\begin{quote}\begin{description}
\sphinxlineitem{Raises}
\sphinxAtStartPar
\sphinxstyleliteralstrong{\sphinxupquote{FileNotFoundError}} \textendash{} Error message if failed to retrieve file path.

\end{description}\end{quote}

\end{fulllineitems}

\index{set\_symptomes() (frise.save.save method)@\spxentry{set\_symptomes()}\spxextra{frise.save.save method}}

\begin{fulllineitems}
\phantomsection\label{\detokenize{frise:frise.save.save.set_symptomes}}
\pysigstartsignatures
\pysiglinewithargsret{\sphinxbfcode{\sphinxupquote{set\_symptomes}}}{\sphinxparam{\DUrole{n}{Liste\_symptomes}}}{}
\pysigstopsignatures
\sphinxAtStartPar
Updates the symptoms.
\begin{quote}\begin{description}
\sphinxlineitem{Parameters}
\sphinxAtStartPar
\sphinxstyleliteralstrong{\sphinxupquote{Liste\_symptomes}} (\sphinxstyleliteralemphasis{\sphinxupquote{list}}) \textendash{} List of symptoms

\end{description}\end{quote}

\end{fulllineitems}

\index{write\_report() (frise.save.save method)@\spxentry{write\_report()}\spxextra{frise.save.save method}}

\begin{fulllineitems}
\phantomsection\label{\detokenize{frise:frise.save.save.write_report}}
\pysigstartsignatures
\pysiglinewithargsret{\sphinxbfcode{\sphinxupquote{write\_report}}}{}{}
\pysigstopsignatures
\sphinxAtStartPar
Writes a human\sphinxhyphen{}readable file.
\begin{quote}\begin{description}
\sphinxlineitem{Raises}
\sphinxAtStartPar
\sphinxstyleliteralstrong{\sphinxupquote{FileNotFoundError}} \textendash{} Error message if failed to retrieve file path.

\end{description}\end{quote}

\end{fulllineitems}


\end{fulllineitems}


\sphinxstepscope


\chapter{annotation package}
\label{\detokenize{annotation:annotation-package}}\label{\detokenize{annotation::doc}}

\section{Submodules}
\label{\detokenize{annotation:submodules}}

\section{annotation.class\_symptome module}
\label{\detokenize{annotation:module-annotation.class_symptome}}\label{\detokenize{annotation:annotation-class-symptome-module}}\index{module@\spxentry{module}!annotation.class\_symptome@\spxentry{annotation.class\_symptome}}\index{annotation.class\_symptome@\spxentry{annotation.class\_symptome}!module@\spxentry{module}}
\sphinxAtStartPar
This module defines a Symptome class used for instantiating symptom objects.
\index{Symptome (class in annotation.class\_symptome)@\spxentry{Symptome}\spxextra{class in annotation.class\_symptome}}

\begin{fulllineitems}
\phantomsection\label{\detokenize{annotation:annotation.class_symptome.Symptome}}
\pysigstartsignatures
\pysiglinewithargsret{\sphinxbfcode{\sphinxupquote{class\DUrole{w}{ }}}\sphinxcode{\sphinxupquote{annotation.class\_symptome.}}\sphinxbfcode{\sphinxupquote{Symptome}}}{\sphinxparam{\DUrole{n}{ID}\DUrole{o}{=}\DUrole{default_value}{None}}\sphinxparamcomma \sphinxparam{\DUrole{n}{Name}\DUrole{o}{=}\DUrole{default_value}{None}}\sphinxparamcomma \sphinxparam{\DUrole{n}{Lateralization}\DUrole{o}{=}\DUrole{default_value}{None}}\sphinxparamcomma \sphinxparam{\DUrole{n}{Topography}\DUrole{o}{=}\DUrole{default_value}{None}}\sphinxparamcomma \sphinxparam{\DUrole{n}{Orientation}\DUrole{o}{=}\DUrole{default_value}{None}}\sphinxparamcomma \sphinxparam{\DUrole{n}{AttributSuppl}\DUrole{o}{=}\DUrole{default_value}{None}}\sphinxparamcomma \sphinxparam{\DUrole{n}{Tdeb}\DUrole{o}{=}\DUrole{default_value}{None}}\sphinxparamcomma \sphinxparam{\DUrole{n}{Tfin}\DUrole{o}{=}\DUrole{default_value}{None}}\sphinxparamcomma \sphinxparam{\DUrole{n}{Comment}\DUrole{o}{=}\DUrole{default_value}{None}}}{}
\pysigstopsignatures
\sphinxAtStartPar
Bases: \sphinxcode{\sphinxupquote{object}}

\sphinxAtStartPar
A class used to instantiate a Symptome object with all its attributes.
\index{ID (annotation.class\_symptome.Symptome attribute)@\spxentry{ID}\spxextra{annotation.class\_symptome.Symptome attribute}}

\begin{fulllineitems}
\phantomsection\label{\detokenize{annotation:annotation.class_symptome.Symptome.ID}}
\pysigstartsignatures
\pysigline{\sphinxbfcode{\sphinxupquote{ID}}}
\pysigstopsignatures
\sphinxAtStartPar
The ID of the symptom.
\begin{quote}\begin{description}
\sphinxlineitem{Type}
\sphinxAtStartPar
str

\end{description}\end{quote}

\end{fulllineitems}

\index{Name (annotation.class\_symptome.Symptome attribute)@\spxentry{Name}\spxextra{annotation.class\_symptome.Symptome attribute}}

\begin{fulllineitems}
\phantomsection\label{\detokenize{annotation:annotation.class_symptome.Symptome.Name}}
\pysigstartsignatures
\pysigline{\sphinxbfcode{\sphinxupquote{Name}}}
\pysigstopsignatures
\sphinxAtStartPar
The name of the symptom.
\begin{quote}\begin{description}
\sphinxlineitem{Type}
\sphinxAtStartPar
str

\end{description}\end{quote}

\end{fulllineitems}

\index{Lateralization (annotation.class\_symptome.Symptome attribute)@\spxentry{Lateralization}\spxextra{annotation.class\_symptome.Symptome attribute}}

\begin{fulllineitems}
\phantomsection\label{\detokenize{annotation:annotation.class_symptome.Symptome.Lateralization}}
\pysigstartsignatures
\pysigline{\sphinxbfcode{\sphinxupquote{Lateralization}}}
\pysigstopsignatures
\sphinxAtStartPar
The lateralization of the symptom.
\begin{quote}\begin{description}
\sphinxlineitem{Type}
\sphinxAtStartPar
str

\end{description}\end{quote}

\end{fulllineitems}

\index{Topography (annotation.class\_symptome.Symptome attribute)@\spxentry{Topography}\spxextra{annotation.class\_symptome.Symptome attribute}}

\begin{fulllineitems}
\phantomsection\label{\detokenize{annotation:annotation.class_symptome.Symptome.Topography}}
\pysigstartsignatures
\pysigline{\sphinxbfcode{\sphinxupquote{Topography}}}
\pysigstopsignatures
\sphinxAtStartPar
The topography of the symptom.
\begin{quote}\begin{description}
\sphinxlineitem{Type}
\sphinxAtStartPar
str

\end{description}\end{quote}

\end{fulllineitems}

\index{Orientation (annotation.class\_symptome.Symptome attribute)@\spxentry{Orientation}\spxextra{annotation.class\_symptome.Symptome attribute}}

\begin{fulllineitems}
\phantomsection\label{\detokenize{annotation:annotation.class_symptome.Symptome.Orientation}}
\pysigstartsignatures
\pysigline{\sphinxbfcode{\sphinxupquote{Orientation}}}
\pysigstopsignatures
\sphinxAtStartPar
The orientation of the symptom.
\begin{quote}\begin{description}
\sphinxlineitem{Type}
\sphinxAtStartPar
str

\end{description}\end{quote}

\end{fulllineitems}

\index{AttributSuppl (annotation.class\_symptome.Symptome attribute)@\spxentry{AttributSuppl}\spxextra{annotation.class\_symptome.Symptome attribute}}

\begin{fulllineitems}
\phantomsection\label{\detokenize{annotation:annotation.class_symptome.Symptome.AttributSuppl}}
\pysigstartsignatures
\pysigline{\sphinxbfcode{\sphinxupquote{AttributSuppl}}}
\pysigstopsignatures
\sphinxAtStartPar
Additional attributes of the symptom.
\begin{quote}\begin{description}
\sphinxlineitem{Type}
\sphinxAtStartPar
str

\end{description}\end{quote}

\end{fulllineitems}

\index{Tdeb (annotation.class\_symptome.Symptome attribute)@\spxentry{Tdeb}\spxextra{annotation.class\_symptome.Symptome attribute}}

\begin{fulllineitems}
\phantomsection\label{\detokenize{annotation:annotation.class_symptome.Symptome.Tdeb}}
\pysigstartsignatures
\pysigline{\sphinxbfcode{\sphinxupquote{Tdeb}}}
\pysigstopsignatures
\sphinxAtStartPar
The start time of the symptom.
\begin{quote}\begin{description}
\sphinxlineitem{Type}
\sphinxAtStartPar
str

\end{description}\end{quote}

\end{fulllineitems}

\index{Tfin (annotation.class\_symptome.Symptome attribute)@\spxentry{Tfin}\spxextra{annotation.class\_symptome.Symptome attribute}}

\begin{fulllineitems}
\phantomsection\label{\detokenize{annotation:annotation.class_symptome.Symptome.Tfin}}
\pysigstartsignatures
\pysigline{\sphinxbfcode{\sphinxupquote{Tfin}}}
\pysigstopsignatures
\sphinxAtStartPar
The end time of the symptom.
\begin{quote}\begin{description}
\sphinxlineitem{Type}
\sphinxAtStartPar
str

\end{description}\end{quote}

\end{fulllineitems}

\index{Comment (annotation.class\_symptome.Symptome attribute)@\spxentry{Comment}\spxextra{annotation.class\_symptome.Symptome attribute}}

\begin{fulllineitems}
\phantomsection\label{\detokenize{annotation:annotation.class_symptome.Symptome.Comment}}
\pysigstartsignatures
\pysigline{\sphinxbfcode{\sphinxupquote{Comment}}}
\pysigstopsignatures
\sphinxAtStartPar
Any additional comments related to the symptom.
\begin{quote}\begin{description}
\sphinxlineitem{Type}
\sphinxAtStartPar
str

\end{description}\end{quote}

\end{fulllineitems}

\index{get\_AttributSuppl() (annotation.class\_symptome.Symptome method)@\spxentry{get\_AttributSuppl()}\spxextra{annotation.class\_symptome.Symptome method}}

\begin{fulllineitems}
\phantomsection\label{\detokenize{annotation:annotation.class_symptome.Symptome.get_AttributSuppl}}
\pysigstartsignatures
\pysiglinewithargsret{\sphinxbfcode{\sphinxupquote{get\_AttributSuppl}}}{}{}
\pysigstopsignatures
\sphinxAtStartPar
Returns the additional attributes of the symptom.

\end{fulllineitems}

\index{get\_Comment() (annotation.class\_symptome.Symptome method)@\spxentry{get\_Comment()}\spxextra{annotation.class\_symptome.Symptome method}}

\begin{fulllineitems}
\phantomsection\label{\detokenize{annotation:annotation.class_symptome.Symptome.get_Comment}}
\pysigstartsignatures
\pysiglinewithargsret{\sphinxbfcode{\sphinxupquote{get\_Comment}}}{}{}
\pysigstopsignatures
\end{fulllineitems}

\index{get\_ID() (annotation.class\_symptome.Symptome method)@\spxentry{get\_ID()}\spxextra{annotation.class\_symptome.Symptome method}}

\begin{fulllineitems}
\phantomsection\label{\detokenize{annotation:annotation.class_symptome.Symptome.get_ID}}
\pysigstartsignatures
\pysiglinewithargsret{\sphinxbfcode{\sphinxupquote{get\_ID}}}{}{}
\pysigstopsignatures
\sphinxAtStartPar
Returns the ID of the symptom.

\end{fulllineitems}

\index{get\_Lateralization() (annotation.class\_symptome.Symptome method)@\spxentry{get\_Lateralization()}\spxextra{annotation.class\_symptome.Symptome method}}

\begin{fulllineitems}
\phantomsection\label{\detokenize{annotation:annotation.class_symptome.Symptome.get_Lateralization}}
\pysigstartsignatures
\pysiglinewithargsret{\sphinxbfcode{\sphinxupquote{get\_Lateralization}}}{}{}
\pysigstopsignatures
\sphinxAtStartPar
Returns the lateralization of the symptom.

\end{fulllineitems}

\index{get\_Name() (annotation.class\_symptome.Symptome method)@\spxentry{get\_Name()}\spxextra{annotation.class\_symptome.Symptome method}}

\begin{fulllineitems}
\phantomsection\label{\detokenize{annotation:annotation.class_symptome.Symptome.get_Name}}
\pysigstartsignatures
\pysiglinewithargsret{\sphinxbfcode{\sphinxupquote{get\_Name}}}{}{}
\pysigstopsignatures
\sphinxAtStartPar
Returns the name of the symptom.

\end{fulllineitems}

\index{get\_Orientation() (annotation.class\_symptome.Symptome method)@\spxentry{get\_Orientation()}\spxextra{annotation.class\_symptome.Symptome method}}

\begin{fulllineitems}
\phantomsection\label{\detokenize{annotation:annotation.class_symptome.Symptome.get_Orientation}}
\pysigstartsignatures
\pysiglinewithargsret{\sphinxbfcode{\sphinxupquote{get\_Orientation}}}{}{}
\pysigstopsignatures
\sphinxAtStartPar
Returns the orientation of the symptom.

\end{fulllineitems}

\index{get\_Tdeb() (annotation.class\_symptome.Symptome method)@\spxentry{get\_Tdeb()}\spxextra{annotation.class\_symptome.Symptome method}}

\begin{fulllineitems}
\phantomsection\label{\detokenize{annotation:annotation.class_symptome.Symptome.get_Tdeb}}
\pysigstartsignatures
\pysiglinewithargsret{\sphinxbfcode{\sphinxupquote{get\_Tdeb}}}{}{}
\pysigstopsignatures
\sphinxAtStartPar
Returns the start time of the symptom.

\end{fulllineitems}

\index{get\_Tfin() (annotation.class\_symptome.Symptome method)@\spxentry{get\_Tfin()}\spxextra{annotation.class\_symptome.Symptome method}}

\begin{fulllineitems}
\phantomsection\label{\detokenize{annotation:annotation.class_symptome.Symptome.get_Tfin}}
\pysigstartsignatures
\pysiglinewithargsret{\sphinxbfcode{\sphinxupquote{get\_Tfin}}}{}{}
\pysigstopsignatures
\sphinxAtStartPar
Returns the end time of the symptom.

\end{fulllineitems}

\index{get\_Topography() (annotation.class\_symptome.Symptome method)@\spxentry{get\_Topography()}\spxextra{annotation.class\_symptome.Symptome method}}

\begin{fulllineitems}
\phantomsection\label{\detokenize{annotation:annotation.class_symptome.Symptome.get_Topography}}
\pysigstartsignatures
\pysiglinewithargsret{\sphinxbfcode{\sphinxupquote{get\_Topography}}}{}{}
\pysigstopsignatures
\sphinxAtStartPar
Returns the topography of the symptom.

\end{fulllineitems}

\index{get\_attributs() (annotation.class\_symptome.Symptome method)@\spxentry{get\_attributs()}\spxextra{annotation.class\_symptome.Symptome method}}

\begin{fulllineitems}
\phantomsection\label{\detokenize{annotation:annotation.class_symptome.Symptome.get_attributs}}
\pysigstartsignatures
\pysiglinewithargsret{\sphinxbfcode{\sphinxupquote{get\_attributs}}}{}{}
\pysigstopsignatures
\sphinxAtStartPar
Returns a list containing the values of the object’s attributes.
\begin{quote}\begin{description}
\sphinxlineitem{Returns}
\sphinxAtStartPar
\begin{description}
\sphinxlineitem{A list containing the values of all attributes.}
\sphinxAtStartPar
{[}
ID,
Name,
Lateralization,
Topography,
Orientation,
AttributSuppl,
Tdeb,
Tfin,
Comment
{]}

\end{description}


\sphinxlineitem{Return type}
\sphinxAtStartPar
\sphinxcode{\sphinxupquote{list}} of \sphinxcode{\sphinxupquote{str}}

\end{description}\end{quote}

\end{fulllineitems}

\index{set\_AttributSuppl() (annotation.class\_symptome.Symptome method)@\spxentry{set\_AttributSuppl()}\spxextra{annotation.class\_symptome.Symptome method}}

\begin{fulllineitems}
\phantomsection\label{\detokenize{annotation:annotation.class_symptome.Symptome.set_AttributSuppl}}
\pysigstartsignatures
\pysiglinewithargsret{\sphinxbfcode{\sphinxupquote{set\_AttributSuppl}}}{\sphinxparam{\DUrole{n}{new\_AttributSuppl}}}{}
\pysigstopsignatures
\sphinxAtStartPar
Sets the additional attributes of the symptom.

\end{fulllineitems}

\index{set\_Comment() (annotation.class\_symptome.Symptome method)@\spxentry{set\_Comment()}\spxextra{annotation.class\_symptome.Symptome method}}

\begin{fulllineitems}
\phantomsection\label{\detokenize{annotation:annotation.class_symptome.Symptome.set_Comment}}
\pysigstartsignatures
\pysiglinewithargsret{\sphinxbfcode{\sphinxupquote{set\_Comment}}}{\sphinxparam{\DUrole{n}{new\_Comment}}}{}
\pysigstopsignatures
\sphinxAtStartPar
Sets any additional comments related to the symptom.

\end{fulllineitems}

\index{set\_ID() (annotation.class\_symptome.Symptome method)@\spxentry{set\_ID()}\spxextra{annotation.class\_symptome.Symptome method}}

\begin{fulllineitems}
\phantomsection\label{\detokenize{annotation:annotation.class_symptome.Symptome.set_ID}}
\pysigstartsignatures
\pysiglinewithargsret{\sphinxbfcode{\sphinxupquote{set\_ID}}}{\sphinxparam{\DUrole{n}{new\_ID}}}{}
\pysigstopsignatures
\sphinxAtStartPar
Sets the ID of the symptom.

\end{fulllineitems}

\index{set\_Lateralization() (annotation.class\_symptome.Symptome method)@\spxentry{set\_Lateralization()}\spxextra{annotation.class\_symptome.Symptome method}}

\begin{fulllineitems}
\phantomsection\label{\detokenize{annotation:annotation.class_symptome.Symptome.set_Lateralization}}
\pysigstartsignatures
\pysiglinewithargsret{\sphinxbfcode{\sphinxupquote{set\_Lateralization}}}{\sphinxparam{\DUrole{n}{new\_Lateralization}}}{}
\pysigstopsignatures
\sphinxAtStartPar
Sets the lateralization of the symptom.

\end{fulllineitems}

\index{set\_Name() (annotation.class\_symptome.Symptome method)@\spxentry{set\_Name()}\spxextra{annotation.class\_symptome.Symptome method}}

\begin{fulllineitems}
\phantomsection\label{\detokenize{annotation:annotation.class_symptome.Symptome.set_Name}}
\pysigstartsignatures
\pysiglinewithargsret{\sphinxbfcode{\sphinxupquote{set\_Name}}}{\sphinxparam{\DUrole{n}{new\_Name}}}{}
\pysigstopsignatures
\sphinxAtStartPar
Sets the name of the symptom.

\end{fulllineitems}

\index{set\_Orientation() (annotation.class\_symptome.Symptome method)@\spxentry{set\_Orientation()}\spxextra{annotation.class\_symptome.Symptome method}}

\begin{fulllineitems}
\phantomsection\label{\detokenize{annotation:annotation.class_symptome.Symptome.set_Orientation}}
\pysigstartsignatures
\pysiglinewithargsret{\sphinxbfcode{\sphinxupquote{set\_Orientation}}}{\sphinxparam{\DUrole{n}{new\_Orientation}}}{}
\pysigstopsignatures
\sphinxAtStartPar
Sets the orientation of the symptom.

\end{fulllineitems}

\index{set\_Tdeb() (annotation.class\_symptome.Symptome method)@\spxentry{set\_Tdeb()}\spxextra{annotation.class\_symptome.Symptome method}}

\begin{fulllineitems}
\phantomsection\label{\detokenize{annotation:annotation.class_symptome.Symptome.set_Tdeb}}
\pysigstartsignatures
\pysiglinewithargsret{\sphinxbfcode{\sphinxupquote{set\_Tdeb}}}{\sphinxparam{\DUrole{n}{new\_Tdeb}}}{}
\pysigstopsignatures
\sphinxAtStartPar
Sets the start time of the symptom.

\end{fulllineitems}

\index{set\_Tfin() (annotation.class\_symptome.Symptome method)@\spxentry{set\_Tfin()}\spxextra{annotation.class\_symptome.Symptome method}}

\begin{fulllineitems}
\phantomsection\label{\detokenize{annotation:annotation.class_symptome.Symptome.set_Tfin}}
\pysigstartsignatures
\pysiglinewithargsret{\sphinxbfcode{\sphinxupquote{set\_Tfin}}}{\sphinxparam{\DUrole{n}{new\_Tfin}}}{}
\pysigstopsignatures
\sphinxAtStartPar
Sets the end time of the symptom.

\end{fulllineitems}

\index{set\_Topography() (annotation.class\_symptome.Symptome method)@\spxentry{set\_Topography()}\spxextra{annotation.class\_symptome.Symptome method}}

\begin{fulllineitems}
\phantomsection\label{\detokenize{annotation:annotation.class_symptome.Symptome.set_Topography}}
\pysigstartsignatures
\pysiglinewithargsret{\sphinxbfcode{\sphinxupquote{set\_Topography}}}{\sphinxparam{\DUrole{n}{new\_Topography}}}{}
\pysigstopsignatures
\sphinxAtStartPar
Sets the topography of the symptom.

\end{fulllineitems}


\end{fulllineitems}



\section{annotation.pop\_up module}
\label{\detokenize{annotation:module-annotation.pop_up}}\label{\detokenize{annotation:annotation-pop-up-module}}\index{module@\spxentry{module}!annotation.pop\_up@\spxentry{annotation.pop\_up}}\index{annotation.pop\_up@\spxentry{annotation.pop\_up}!module@\spxentry{module}}
\sphinxAtStartPar
This module defines a SymptomeEditor class for editing symptom attributes via a graphical user interface.
\index{SymptomeEditor (class in annotation.pop\_up)@\spxentry{SymptomeEditor}\spxextra{class in annotation.pop\_up}}

\begin{fulllineitems}
\phantomsection\label{\detokenize{annotation:annotation.pop_up.SymptomeEditor}}
\pysigstartsignatures
\pysiglinewithargsret{\sphinxbfcode{\sphinxupquote{class\DUrole{w}{ }}}\sphinxcode{\sphinxupquote{annotation.pop\_up.}}\sphinxbfcode{\sphinxupquote{SymptomeEditor}}}{\sphinxparam{\DUrole{n}{Symp}}}{}
\pysigstopsignatures
\sphinxAtStartPar
Bases: \sphinxcode{\sphinxupquote{CTkToplevel}}

\sphinxAtStartPar
SymptomeEditor class for editing symptom attributes via a graphical user interface.
Allows editing of symptoms through a window.
\index{Symp (annotation.pop\_up.SymptomeEditor attribute)@\spxentry{Symp}\spxextra{annotation.pop\_up.SymptomeEditor attribute}}

\begin{fulllineitems}
\phantomsection\label{\detokenize{annotation:annotation.pop_up.SymptomeEditor.Symp}}
\pysigstartsignatures
\pysigline{\sphinxbfcode{\sphinxupquote{Symp}}}
\pysigstopsignatures
\sphinxAtStartPar
An instance of the Symptome class representing the symptom to be edited.
\begin{quote}\begin{description}
\sphinxlineitem{Type}
\sphinxAtStartPar
\sphinxcode{\sphinxupquote{Symptome}}

\end{description}\end{quote}

\end{fulllineitems}

\index{apply\_changes() (annotation.pop\_up.SymptomeEditor method)@\spxentry{apply\_changes()}\spxextra{annotation.pop\_up.SymptomeEditor method}}

\begin{fulllineitems}
\phantomsection\label{\detokenize{annotation:annotation.pop_up.SymptomeEditor.apply_changes}}
\pysigstartsignatures
\pysiglinewithargsret{\sphinxbfcode{\sphinxupquote{apply\_changes}}}{\sphinxparam{\DUrole{n}{Symp}}}{}
\pysigstopsignatures
\sphinxAtStartPar
Retrieves the entered values and updates the attributes of the Symp class.
\begin{quote}\begin{description}
\sphinxlineitem{Parameters}
\sphinxAtStartPar
\sphinxstyleliteralstrong{\sphinxupquote{Symp}} (\sphinxcode{\sphinxupquote{Symptome}}) \textendash{} The Symptome object to be updated.

\end{description}\end{quote}

\end{fulllineitems}

\index{create\_entry() (annotation.pop\_up.SymptomeEditor method)@\spxentry{create\_entry()}\spxextra{annotation.pop\_up.SymptomeEditor method}}

\begin{fulllineitems}
\phantomsection\label{\detokenize{annotation:annotation.pop_up.SymptomeEditor.create_entry}}
\pysigstartsignatures
\pysiglinewithargsret{\sphinxbfcode{\sphinxupquote{create\_entry}}}{\sphinxparam{\DUrole{n}{label\_text}}\sphinxparamcomma \sphinxparam{\DUrole{n}{initial\_value}}\sphinxparamcomma \sphinxparam{\DUrole{n}{i}}}{}
\pysigstopsignatures
\sphinxAtStartPar
Creates and places text entry fields (entries) on the window.
\begin{quote}\begin{description}
\sphinxlineitem{Parameters}\begin{itemize}
\item {} 
\sphinxAtStartPar
\sphinxstyleliteralstrong{\sphinxupquote{label\_text}} (\sphinxstyleliteralemphasis{\sphinxupquote{str}}) \textendash{} The label for the entry.

\item {} 
\sphinxAtStartPar
\sphinxstyleliteralstrong{\sphinxupquote{initial\_value}} (\sphinxstyleliteralemphasis{\sphinxupquote{str}}) \textendash{} The current value of the entry.

\item {} 
\sphinxAtStartPar
\sphinxstyleliteralstrong{\sphinxupquote{i}} (\sphinxstyleliteralemphasis{\sphinxupquote{int}}) \textendash{} The index of the entry row.

\end{itemize}

\end{description}\end{quote}

\end{fulllineitems}


\end{fulllineitems}



\section{annotation.load\_symptomes module}
\label{\detokenize{annotation:module-annotation.load_symptomes}}\label{\detokenize{annotation:annotation-load-symptomes-module}}\index{module@\spxentry{module}!annotation.load\_symptomes@\spxentry{annotation.load\_symptomes}}\index{annotation.load\_symptomes@\spxentry{annotation.load\_symptomes}!module@\spxentry{module}}
\sphinxAtStartPar
Functions for loading symptoms from text files
\index{read\_symptoms() (in module annotation.load\_symptomes)@\spxentry{read\_symptoms()}\spxextra{in module annotation.load\_symptomes}}

\begin{fulllineitems}
\phantomsection\label{\detokenize{annotation:annotation.load_symptomes.read_symptoms}}
\pysigstartsignatures
\pysiglinewithargsret{\sphinxcode{\sphinxupquote{annotation.load\_symptomes.}}\sphinxbfcode{\sphinxupquote{read\_symptoms}}}{\sphinxparam{\DUrole{n}{filename}}}{}
\pysigstopsignatures
\sphinxAtStartPar
Reads a text file containing symptoms and returns a list of Symptome objects.
\begin{quote}\begin{description}
\sphinxlineitem{Parameters}
\sphinxAtStartPar
\sphinxstyleliteralstrong{\sphinxupquote{filename}} (\sphinxstyleliteralemphasis{\sphinxupquote{str}}) \textendash{} The path to the text file containing symptoms.

\sphinxlineitem{Returns}
\sphinxAtStartPar
A list of Symptome objects.

\sphinxlineitem{Return type}
\sphinxAtStartPar
list

\end{description}\end{quote}

\end{fulllineitems}



\section{annotation.exel\_menus module}
\label{\detokenize{annotation:module-annotation.exel_menus}}\label{\detokenize{annotation:annotation-exel-menus-module}}\index{module@\spxentry{module}!annotation.exel\_menus@\spxentry{annotation.exel\_menus}}\index{annotation.exel\_menus@\spxentry{annotation.exel\_menus}!module@\spxentry{module}}
\sphinxAtStartPar
menus and submenus based on exel file
\index{Read\_excel() (in module annotation.exel\_menus)@\spxentry{Read\_excel()}\spxextra{in module annotation.exel\_menus}}

\begin{fulllineitems}
\phantomsection\label{\detokenize{annotation:annotation.exel_menus.Read_excel}}
\pysigstartsignatures
\pysiglinewithargsret{\sphinxcode{\sphinxupquote{annotation.exel\_menus.}}\sphinxbfcode{\sphinxupquote{Read\_excel}}}{\sphinxparam{\DUrole{n}{file\_path}}}{}
\pysigstopsignatures
\sphinxAtStartPar
Reads an Excel file and organizes the data into a hierarchical structure for a graphical interface.
\begin{quote}\begin{description}
\sphinxlineitem{Parameters}
\sphinxAtStartPar
\sphinxstyleliteralstrong{\sphinxupquote{file\_path}} (\sphinxstyleliteralemphasis{\sphinxupquote{str}}) \textendash{} Path to the Excel file containing the data.

\sphinxlineitem{Returns}
\sphinxAtStartPar
A hierarchical dictionary containing organized symptom data.

\sphinxlineitem{Return type}
\sphinxAtStartPar
dict

\end{description}\end{quote}

\sphinxAtStartPar
The function reads an Excel file specified by ‘file\_path’,
processes the data, and organizes it into a hierarchical structure.
This structure is designed to facilitate the creation of a graphical
interface for displaying and interacting with the symptom data.

\sphinxAtStartPar
The Excel file is expected to have columns named ‘Typologie’, ‘Designation’,
‘Description’, ‘Sub\_description’,’Topography’, and ‘Lateralized’.
\begin{description}
\sphinxlineitem{The data is organized based on the following criteria:}\begin{itemize}
\item {} 
\sphinxAtStartPar
There can not be sub sescription without description.

\item {} 
\sphinxAtStartPar
If there is no description, no sub\_description no topography, but there are lateralizations, they are stored under an empty string.

\item {} 
\sphinxAtStartPar
If there is no additional information, a ‘None’ flag is stored.

\item {} 
\sphinxAtStartPar
If there is topography and lateralizations but no description, the data is stored based on topography.

\item {} 
\sphinxAtStartPar
Otherwise, the detailed information is stored including description, topography, and lateralizations.

\end{itemize}

\end{description}
\subsubsection*{Example}

\begin{sphinxVerbatim}[commandchars=\\\{\}]
\PYG{g+gp}{\PYGZgt{}\PYGZgt{}\PYGZgt{} }\PYG{n}{Read\PYGZus{}excel}\PYG{p}{(}\PYG{l+s+s1}{\PYGZsq{}}\PYG{l+s+s1}{ictal\PYGZus{}symptoms.xlsx}\PYG{l+s+s1}{\PYGZsq{}}\PYG{p}{)}
\end{sphinxVerbatim}

\begin{sphinxadmonition}{note}{Note:}
\sphinxAtStartPar
The function uses Pandas to read the Excel file and defaultdict
to create the hierarchical structure.
\end{sphinxadmonition}

\end{fulllineitems}

\index{add\_submenus() (in module annotation.exel\_menus)@\spxentry{add\_submenus()}\spxextra{in module annotation.exel\_menus}}

\begin{fulllineitems}
\phantomsection\label{\detokenize{annotation:annotation.exel_menus.add_submenus}}
\pysigstartsignatures
\pysiglinewithargsret{\sphinxcode{\sphinxupquote{annotation.exel\_menus.}}\sphinxbfcode{\sphinxupquote{add\_submenus}}}{\sphinxparam{\DUrole{n}{menu}}\sphinxparamcomma \sphinxparam{\DUrole{n}{data}}\sphinxparamcomma \sphinxparam{\DUrole{n}{full\_path}}\sphinxparamcomma \sphinxparam{\DUrole{n}{on\_select}}}{}
\pysigstopsignatures
\sphinxAtStartPar
Add submenus to the main menu based on the provided data.

\sphinxAtStartPar
This function iterates through the data dictionary to add submenus
and menu items to the provided Tkinter menu. It checks for different
scenarios in the data structure to determine the appropriate way to
add submenus and menu items.
\begin{quote}\begin{description}
\sphinxlineitem{Parameters}\begin{itemize}
\item {} 
\sphinxAtStartPar
\sphinxstyleliteralstrong{\sphinxupquote{menu}} (\sphinxstyleliteralemphasis{\sphinxupquote{tk.Menu}}) \textendash{} The menu to which the submenus will be added.

\item {} 
\sphinxAtStartPar
\sphinxstyleliteralstrong{\sphinxupquote{data}} (\sphinxstyleliteralemphasis{\sphinxupquote{dict}}) \textendash{} The hierarchical data containing submenu information.

\item {} 
\sphinxAtStartPar
\sphinxstyleliteralstrong{\sphinxupquote{full\_path}} (\sphinxstyleliteralemphasis{\sphinxupquote{str}}) \textendash{} The full path to the current menu item.

\item {} 
\sphinxAtStartPar
\sphinxstyleliteralstrong{\sphinxupquote{on\_select}} (\sphinxstyleliteralemphasis{\sphinxupquote{function}}) \textendash{} The function to select the symptom

\end{itemize}

\end{description}\end{quote}

\end{fulllineitems}

\index{build\_menu() (in module annotation.exel\_menus)@\spxentry{build\_menu()}\spxextra{in module annotation.exel\_menus}}

\begin{fulllineitems}
\phantomsection\label{\detokenize{annotation:annotation.exel_menus.build_menu}}
\pysigstartsignatures
\pysiglinewithargsret{\sphinxcode{\sphinxupquote{annotation.exel\_menus.}}\sphinxbfcode{\sphinxupquote{build\_menu}}}{\sphinxparam{\DUrole{n}{structure}}\sphinxparamcomma \sphinxparam{\DUrole{n}{main\_menu}}\sphinxparamcomma \sphinxparam{\DUrole{n}{on\_select}}}{}
\pysigstopsignatures
\sphinxAtStartPar
Build a cascading menu based on the provided structure.

\sphinxAtStartPar
This function builds a cascading menu using Tkinter based on the
hierarchical structure provided. It iterates through the structure
to add main menu items, submenus, and menu items. Depending on the
data in the structure, it calls the add\_submenus function to add
appropriate submenus and menu items to the main menu.
\begin{quote}\begin{description}
\sphinxlineitem{Parameters}\begin{itemize}
\item {} 
\sphinxAtStartPar
\sphinxstyleliteralstrong{\sphinxupquote{structure}} (\sphinxstyleliteralemphasis{\sphinxupquote{dict}}) \textendash{} The hierarchical structure of the menu.

\item {} 
\sphinxAtStartPar
\sphinxstyleliteralstrong{\sphinxupquote{main\_menu}} (\sphinxstyleliteralemphasis{\sphinxupquote{tk.Menu}}) \textendash{} The main menu to which the cascading menu will be added.

\item {} 
\sphinxAtStartPar
\sphinxstyleliteralstrong{\sphinxupquote{on\_select}} (\sphinxstyleliteralemphasis{\sphinxupquote{function}}) \textendash{} The function to select the symptom

\end{itemize}

\end{description}\end{quote}

\end{fulllineitems}



\section{annotation.search\_bar}
\label{\detokenize{annotation:module-annotation.search_bar}}\label{\detokenize{annotation:annotation-search-bar}}\index{module@\spxentry{module}!annotation.search\_bar@\spxentry{annotation.search\_bar}}\index{annotation.search\_bar@\spxentry{annotation.search\_bar}!module@\spxentry{module}}\index{search\_symptomes (class in annotation.search\_bar)@\spxentry{search\_symptomes}\spxextra{class in annotation.search\_bar}}

\begin{fulllineitems}
\phantomsection\label{\detokenize{annotation:annotation.search_bar.search_symptomes}}
\pysigstartsignatures
\pysiglinewithargsret{\sphinxbfcode{\sphinxupquote{class\DUrole{w}{ }}}\sphinxcode{\sphinxupquote{annotation.search\_bar.}}\sphinxbfcode{\sphinxupquote{search\_symptomes}}}{\sphinxparam{\DUrole{n}{root}}\sphinxparamcomma \sphinxparam{\DUrole{n}{symptoms\_struct}}\sphinxparamcomma \sphinxparam{\DUrole{n}{myfunc}}\sphinxparamcomma \sphinxparam{\DUrole{n}{width}}}{}
\pysigstopsignatures
\sphinxAtStartPar
Bases: \sphinxcode{\sphinxupquote{object}}

\sphinxAtStartPar
A class to create a symptom search bar in the EpiScope GUI.

\sphinxAtStartPar
This class provides a search bar to look for symptoms in a hierarchical dictionary structure
and display the matching results in a listbox. It also allows selecting an item from the listbox
by clicking on it, triggering a specified function with the selected item as an argument.
\index{symptoms\_structure (annotation.search\_bar.search\_symptomes attribute)@\spxentry{symptoms\_structure}\spxextra{annotation.search\_bar.search\_symptomes attribute}}

\begin{fulllineitems}
\phantomsection\label{\detokenize{annotation:annotation.search_bar.search_symptomes.symptoms_structure}}
\pysigstartsignatures
\pysigline{\sphinxbfcode{\sphinxupquote{symptoms\_structure}}}
\pysigstopsignatures
\sphinxAtStartPar
Hierarchical dictionary containing organized symptom data.
\begin{quote}\begin{description}
\sphinxlineitem{Type}
\sphinxAtStartPar
dict

\end{description}\end{quote}

\end{fulllineitems}

\index{search\_entry (annotation.search\_bar.search\_symptomes attribute)@\spxentry{search\_entry}\spxextra{annotation.search\_bar.search\_symptomes attribute}}

\begin{fulllineitems}
\phantomsection\label{\detokenize{annotation:annotation.search_bar.search_symptomes.search_entry}}
\pysigstartsignatures
\pysigline{\sphinxbfcode{\sphinxupquote{search\_entry}}}
\pysigstopsignatures
\sphinxAtStartPar
Entry widget for user input.
\begin{quote}\begin{description}
\sphinxlineitem{Type}
\sphinxAtStartPar
CTkEntry

\end{description}\end{quote}

\end{fulllineitems}

\index{symptom\_listbox (annotation.search\_bar.search\_symptomes attribute)@\spxentry{symptom\_listbox}\spxextra{annotation.search\_bar.search\_symptomes attribute}}

\begin{fulllineitems}
\phantomsection\label{\detokenize{annotation:annotation.search_bar.search_symptomes.symptom_listbox}}
\pysigstartsignatures
\pysigline{\sphinxbfcode{\sphinxupquote{symptom\_listbox}}}
\pysigstopsignatures
\sphinxAtStartPar
Listbox widget to display search results.
\begin{quote}\begin{description}
\sphinxlineitem{Type}
\sphinxAtStartPar
Listbox

\end{description}\end{quote}

\end{fulllineitems}

\index{search\_element() (annotation.search\_bar.search\_symptomes method)@\spxentry{search\_element()}\spxextra{annotation.search\_bar.search\_symptomes method}}

\begin{fulllineitems}
\phantomsection\label{\detokenize{annotation:annotation.search_bar.search_symptomes.search_element}}
\pysigstartsignatures
\pysiglinewithargsret{\sphinxbfcode{\sphinxupquote{search\_element}}}{\sphinxparam{\DUrole{n}{search\_term}}}{}
\pysigstopsignatures
\sphinxAtStartPar
Searches for matching elements based on the search term.

\end{fulllineitems}

\index{get\_selected() (annotation.search\_bar.search\_symptomes method)@\spxentry{get\_selected()}\spxextra{annotation.search\_bar.search\_symptomes method}}

\begin{fulllineitems}
\phantomsection\label{\detokenize{annotation:annotation.search_bar.search_symptomes.get_selected}}
\pysigstartsignatures
\pysiglinewithargsret{\sphinxbfcode{\sphinxupquote{get\_selected}}}{\sphinxparam{\DUrole{n}{event}}\sphinxparamcomma \sphinxparam{\DUrole{n}{func}}}{}
\pysigstopsignatures
\sphinxAtStartPar
Retrieves the selected item from the listbox.

\end{fulllineitems}

\index{update\_search() (annotation.search\_bar.search\_symptomes method)@\spxentry{update\_search()}\spxextra{annotation.search\_bar.search\_symptomes method}}

\begin{fulllineitems}
\phantomsection\label{\detokenize{annotation:annotation.search_bar.search_symptomes.update_search}}
\pysigstartsignatures
\pysiglinewithargsret{\sphinxbfcode{\sphinxupquote{update\_search}}}{\sphinxparam{\DUrole{n}{event}}}{}
\pysigstopsignatures
\sphinxAtStartPar
Updates the listbox based on the search input.

\end{fulllineitems}

\index{get\_selected() (annotation.search\_bar.search\_symptomes method)@\spxentry{get\_selected()}\spxextra{annotation.search\_bar.search\_symptomes method}}

\begin{fulllineitems}
\phantomsection\label{\detokenize{annotation:id0}}
\pysigstartsignatures
\pysiglinewithargsret{\sphinxbfcode{\sphinxupquote{get\_selected}}}{\sphinxparam{\DUrole{n}{event}}\sphinxparamcomma \sphinxparam{\DUrole{n}{func}}}{}
\pysigstopsignatures
\sphinxAtStartPar
Retrieves the selected item from the listbox and calls the specified function.
\begin{quote}\begin{description}
\sphinxlineitem{Parameters}\begin{itemize}
\item {} 
\sphinxAtStartPar
\sphinxstyleliteralstrong{\sphinxupquote{event}} \textendash{} The event object (not used directly, but required by Tkinter).

\item {} 
\sphinxAtStartPar
\sphinxstyleliteralstrong{\sphinxupquote{func}} (\sphinxstyleliteralemphasis{\sphinxupquote{function}}) \textendash{} Function to be called with the selected item as an argument.

\end{itemize}

\end{description}\end{quote}

\end{fulllineitems}

\index{search\_element() (annotation.search\_bar.search\_symptomes method)@\spxentry{search\_element()}\spxextra{annotation.search\_bar.search\_symptomes method}}

\begin{fulllineitems}
\phantomsection\label{\detokenize{annotation:id1}}
\pysigstartsignatures
\pysiglinewithargsret{\sphinxbfcode{\sphinxupquote{search\_element}}}{\sphinxparam{\DUrole{n}{search\_term}}}{}
\pysigstopsignatures
\sphinxAtStartPar
Searches for elements matching the search term in the ‘Designation’ and ‘Description’ columns.
\begin{quote}\begin{description}
\sphinxlineitem{Parameters}\begin{itemize}
\item {} 
\sphinxAtStartPar
\sphinxstyleliteralstrong{\sphinxupquote{symptoms\_structure}} (\sphinxstyleliteralemphasis{\sphinxupquote{dict}}) \textendash{} Hierarchical dictionary containing organized symptom data.

\item {} 
\sphinxAtStartPar
\sphinxstyleliteralstrong{\sphinxupquote{search\_term}} (\sphinxstyleliteralemphasis{\sphinxupquote{str}}) \textendash{} Term to search for in ‘Designation’ and ‘Description’ columns.

\end{itemize}

\sphinxlineitem{Returns}
\sphinxAtStartPar
List of matching elements found for the search term.

\sphinxlineitem{Return type}
\sphinxAtStartPar
list

\end{description}\end{quote}

\sphinxAtStartPar
This function searches for the specified term in the ‘Designation’ and ‘Description’ columns
of the hierarchical structure and returns a list of matching elements.

\end{fulllineitems}

\index{update\_search() (annotation.search\_bar.search\_symptomes method)@\spxentry{update\_search()}\spxextra{annotation.search\_bar.search\_symptomes method}}

\begin{fulllineitems}
\phantomsection\label{\detokenize{annotation:id2}}
\pysigstartsignatures
\pysiglinewithargsret{\sphinxbfcode{\sphinxupquote{update\_search}}}{\sphinxparam{\DUrole{n}{event}}}{}
\pysigstopsignatures
\sphinxAtStartPar
Update the displayed symptoms based on the user input in the search bar.

\sphinxAtStartPar
This function filters the symptoms based on the user input in the
search bar and updates the listbox to show only the matching symptoms.
\begin{quote}\begin{description}
\sphinxlineitem{Parameters}
\sphinxAtStartPar
\sphinxstyleliteralstrong{\sphinxupquote{event}} \textendash{} The event object (not used in this function, but required by Tkinter)

\end{description}\end{quote}

\end{fulllineitems}


\end{fulllineitems}



\chapter{Indices and tables}
\label{\detokenize{index:indices-and-tables}}\begin{itemize}
\item {} 
\sphinxAtStartPar
\DUrole{xref,std,std-ref}{genindex}

\item {} 
\sphinxAtStartPar
\DUrole{xref,std,std-ref}{modindex}

\item {} 
\sphinxAtStartPar
\DUrole{xref,std,std-ref}{search}

\end{itemize}


\renewcommand{\indexname}{Python Module Index}
\begin{sphinxtheindex}
\let\bigletter\sphinxstyleindexlettergroup
\bigletter{a}
\item\relax\sphinxstyleindexentry{annotation.class\_symptome}\sphinxstyleindexpageref{annotation:\detokenize{module-annotation.class_symptome}}
\item\relax\sphinxstyleindexentry{annotation.exel\_menus}\sphinxstyleindexpageref{annotation:\detokenize{module-annotation.exel_menus}}
\item\relax\sphinxstyleindexentry{annotation.load\_symptomes}\sphinxstyleindexpageref{annotation:\detokenize{module-annotation.load_symptomes}}
\item\relax\sphinxstyleindexentry{annotation.pop\_up}\sphinxstyleindexpageref{annotation:\detokenize{module-annotation.pop_up}}
\item\relax\sphinxstyleindexentry{annotation.search\_bar}\sphinxstyleindexpageref{annotation:\detokenize{module-annotation.search_bar}}
\indexspace
\bigletter{f}
\item\relax\sphinxstyleindexentry{frise.ecriture\_fichier}\sphinxstyleindexpageref{frise:\detokenize{module-frise.ecriture_fichier}}
\item\relax\sphinxstyleindexentry{frise.fonctions\_frise}\sphinxstyleindexpageref{frise:\detokenize{module-frise.fonctions_frise}}
\item\relax\sphinxstyleindexentry{frise.save}\sphinxstyleindexpageref{frise:\detokenize{module-frise.save}}
\indexspace
\bigletter{g}
\item\relax\sphinxstyleindexentry{general\_interface}\sphinxstyleindexpageref{general_interface:\detokenize{module-general_interface}}
\end{sphinxtheindex}

\renewcommand{\indexname}{Index}
\printindex
\end{document}